\documentclass[12pt,a4paper]{article}

% --- Paquetes ---
\usepackage[utf8]{inputenc}
\usepackage[T1]{fontenc}
\usepackage[spanish,es-nodecimaldot]{babel} 
\usepackage{graphicx} 
\usepackage{hyperref} 
\usepackage{geometry} 
\usepackage{fancyhdr} 
\usepackage{color} 
\usepackage{lipsum}
\usepackage{amsmath}
\usepackage{amsfonts}
\usepackage{booktabs}
\usepackage{array}
\usepackage{listings}
\usepackage{xcolor}
\usepackage{fancyvrb}
\usepackage{tikz}
\usepackage{pgfplots}
\usepackage{pgfplotstable}
\usepackage{makeidx}
\usepackage{multicol}
\usepackage{float}
\usetikzlibrary{shapes,arrows,positioning,fit,backgrounds}
\pgfplotsset{compat=1.18}

% --- Configuración del índice ---
\makeindex

% --- Configuración para evitar conflictos con babel español ---
\tikzset{
    every picture/.append style={
        execute at begin picture={\shorthandoff{>}\shorthandoff{<}},
        execute at end picture={\shorthandon{>}\shorthandon{<}}
    }
} 

% --- Márgenes ---
\geometry{
    top=2.5cm,
    bottom=2.5cm,
    left=2.5cm,
    right=2.5cm
}

% --- Configuración de código ---
\lstset{
    basicstyle=\ttfamily\footnotesize,
    breaklines=true,
    breakatwhitespace=true,
    frame=single,
    numbers=left,
    numberstyle=\tiny,
    showstringspaces=false,
    tabsize=2,
    captionpos=b,
    columns=flexible,
    keepspaces=true,
    xleftmargin=0.5cm,
    xrightmargin=0.5cm
}

% Configuración para verbatim con ajuste de línea
\DefineVerbatimEnvironment{Verbatim}{Verbatim}{
    fontsize=\footnotesize,
    breaklines=true,
    breakanywhere=true,
    frame=single,
    framesep=3mm,
    xleftmargin=0.5cm,
    xrightmargin=0.5cm
}

% --- Encabezado y pie de página ---
\pagestyle{fancy}
\fancyhf{}
\fancyhead[L]{DevOps}
\fancyhead[R]{Brandon Pérez Reyes}
\fancyfoot[C]{\thepage}

% --- Datos del documento ---
\title{\textbf{DevOps: Integración y Entrega Continua}}
\author{Brandon Pérez Reyes}
\date{October 2025}

\begin{document}

% --- Portada ---
\begin{titlepage}
    \centering
    \vspace*{2cm}
    {\Huge \bfseries DevOps: Integración y Entrega Continua \par}
    \vspace{2cm}
    % Logo DevOps creado con TikZ
    \begin{tikzpicture}[scale=0.8]
        % Círculo exterior
        \draw[thick, blue!70] (0,0) circle (2cm);
        
        % Símbolo infinito (DevOps loop)
        \draw[very thick, green!60, line width=3pt] 
            (-1.2,0) .. controls (-0.6,0.8) and (0.6,0.8) .. (0,0)
            .. controls (-0.6,-0.8) and (0.6,-0.8) .. (1.2,0);
        
        % Texto DEV
        \node[font=\large\bfseries, blue!80] at (-0.8,0.3) {DEV};
        
        % Texto OPS  
        \node[font=\large\bfseries, orange!80] at (0.8,-0.3) {OPS};
        
        % Engranajes pequeños
        \node[font=\footnotesize] at (-1.5,1.2) {⚙};
        \node[font=\footnotesize] at (1.5,-1.2) {⚙};
        \node[font=\footnotesize] at (1.5,1.2) {⚙};
        \node[font=\footnotesize] at (-1.5,-1.2) {⚙};
    \end{tikzpicture}
    \par
    \vspace{2cm}
    {\Large Ing. Brandon Pérez Reyes \par}
    \vfill
    {\large October 2025 \par}
\end{titlepage}

\newpage

% --- Resumen ---
\begin{abstract}
El presente artículo explora el concepto de DevOps, una metodología que integra el desarrollo de software y las operaciones de TI para lograr una entrega más rápida, confiable y eficiente de aplicaciones. Se describen sus principios, herramientas más utilizadas y beneficios en entornos modernos de desarrollo.
\end{abstract}

% --- Tabla de contenidos ---
\tableofcontents
\newpage

% --- Contenido ---
\section{Introducción}
DevOps\index{DevOps} es un enfoque que combina \textbf{Desarrollo} (Dev) y \textbf{Operaciones} (Ops) para mejorar la colaboración entre equipos y acelerar el ciclo de vida del software. Su principal objetivo es implementar cambios con rapidez, confiabilidad y seguridad, fomentando la integración\index{Integración Continua} y la automatización\index{Automatización} de procesos.

\subsection{Historia y Evolución}
El término DevOps fue acuñado por Patrick Debois en 2009, surgiendo de la necesidad de resolver los conflictos tradicionales entre equipos de desarrollo y operaciones. La evolución histórica incluye:

\begin{itemize}
    \item \textbf{2007-2008:} Primeras discusiones sobre "Agile Infrastructure"
    \item \textbf{2009:} Primera conferencia DevOpsDays en Bélgica
    \item \textbf{2010-2013:} Adopción temprana por empresas como Flickr y Etsy
    \item \textbf{2014-2018:} Mainstream adoption con herramientas como Docker\index{Docker} y Kubernetes\index{Kubernetes}
    \item \textbf{2019-presente:} Evolución hacia DevSecOps\index{DevSecOps} y GitOps\index{GitOps}
\end{itemize}

\subsection{Problemática Tradicional}
Antes de DevOps, las organizaciones enfrentaban:
\begin{itemize}
    \item \textbf{Silos organizacionales:} Equipos aislados con objetivos conflictivos
    \item \textbf{Despliegues lentos:} Procesos manuales propensos a errores
    \item \textbf{Falta de visibilidad:} Monitoreo limitado de aplicaciones en producción
    \item \textbf{Cultura de culpa:} Responsabilidades difusas ante fallos del sistema
\end{itemize}

\section{Principios de DevOps}
Los principios clave de DevOps incluyen:
\begin{itemize}
    \item \textbf{Colaboración\index{Colaboración}:} Integración efectiva entre desarrolladores y operaciones.
    \item \textbf{Automatización\index{Automatización}:} Uso de herramientas para despliegue, pruebas y monitoreo.
    \item \textbf{Entrega continua\index{Entrega Continua}:} Liberación frecuente y confiable de software.
    \item \textbf{Monitoreo constante\index{Monitoreo}:} Supervisión de sistemas para retroalimentación rápida.
    \item \textbf{Cultura de mejora continua\index{Mejora Continua}:} Aprendizaje constante y adaptación.
\end{itemize}

\subsection{Los Tres Caminos de DevOps}
Según Gene Kim, DevOps se basa en tres principios fundamentales:

\begin{enumerate}
    \item \textbf{Primer Camino - Flujo:} Optimizar el flujo de trabajo desde desarrollo hasta operaciones
    \item \textbf{Segundo Camino - Retroalimentación:} Crear bucles de retroalimentación rápidos
    \item \textbf{Tercer Camino - Mejora Continua:} Fomentar una cultura de experimentación y aprendizaje
\end{enumerate}

\subsection{Diagrama: Arquitectura DevOps Completa}
\begin{figure}[h!]
\centering
\begin{tikzpicture}[
    node distance=2.5cm,
    box/.style={rectangle, draw=black!80, fill=blue!30, text width=2.5cm, text centered, minimum height=1.5cm, font=\footnotesize, drop shadow},
    arrow/.style={->, thick, blue!80, line width=2pt},
    feedback/.style={<->, thick, red!70, dashed, line width=2pt}
]

% Development side (left column)
\node[box, fill=green!20] (dev) {Desarrollo\\(Dev)};
\node[box, below=of dev] (code) {Código\\Fuente};
\node[box, below=of code] (ci) {CI Pipeline\\Build/Test};

% Operations side (right column)
\node[box, right=6cm of dev, fill=orange!20] (ops) {Operaciones\\(Ops)};
\node[box, below=of ops] (infra) {Infraestructura\\IaC};
\node[box, below=of infra] (monitor) {Monitoreo\\Alertas};

% Center - Integration
\node[box, right=3cm of code, fill=yellow!20] (collab) {Colaboración\\DevOps};
\node[box, below=of collab, fill=purple!20] (cd) {CD Pipeline\\Deploy};

% Main flow arrows
\draw[arrow] (dev) -- (code);
\draw[arrow] (code) -- (ci);
\draw[arrow] (ci) -- (cd);
\draw[arrow] (cd) -- (infra);
\draw[arrow] (infra) -- (ops);

% Collaboration connections
\draw[arrow, dashed, blue] (dev) -- (collab);
\draw[arrow, dashed, blue] (ops) -- (collab);
\draw[arrow] (collab) -- (cd);

% Feedback loops
\draw[feedback] (monitor) to[bend right=20] (ci);
\draw[feedback] (ops) to[bend left=30] (dev);

% Side labels
\node[left=1.5cm of dev, font=\small, text=green!70!black, rotate=90, anchor=center] {DESARROLLO};
\node[right=1.5cm of ops, font=\small, text=orange!70!black, rotate=90, anchor=center] {OPERACIONES};

% Title (positioned at absolute top center)
\node at (3,6) [font=\Large\bfseries] {Arquitectura DevOps};

% Bottom label
\node[below=1cm of cd, font=\small, text=purple!70!black] {Integración Continua};

\end{tikzpicture}
\caption{Arquitectura DevOps: Integración entre Desarrollo y Operaciones}
\end{figure}


\section{Arquitectura y Patrones DevOps}

\subsection{Arquitectura de Microservicios}
Los microservicios\index{Microservicios} son fundamentales en DevOps, permitiendo:
\begin{itemize}
    \item \textbf{Despliegues independientes:} Cada servicio puede desplegarse por separado
    \item \textbf{Escalabilidad granular:} Escalar solo los componentes necesarios
    \item \textbf{Tecnologías heterogéneas:} Diferentes stacks por servicio
    \item \textbf{Equipos autónomos:} Ownership completo del ciclo de vida
\end{itemize}

\newpage
\subsection{Diagrama: Arquitectura de Microservicios}
\begin{figure}[h!]
\centering
\begin{tikzpicture}[
    service/.style={rectangle, draw, fill=blue!20, text width=2cm, text centered, minimum height=1cm, font=\footnotesize},
    gateway/.style={rectangle, draw, fill=green!30, text width=3cm, text centered, minimum height=1cm},
    database/.style={cylinder, draw, fill=yellow!20, text width=1.5cm, text centered, minimum height=0.8cm, font=\tiny},
    user/.style={circle, draw, fill=purple!20, text width=1cm, text centered, font=\tiny}
]

% Users
\node[user] (mobile) at (0,4) {Mobile\\App};
\node[user] (web) at (0,2) {Web\\App};

% API Gateway
\node[gateway] (gateway) at (3,3) {API Gateway\\Load Balancer};

% Microservices
\node[service] (auth) at (6,5) {Auth\\Service};
\node[service] (user-svc) at (6,3.5) {User\\Service};
\node[service] (order) at (6,2) {Order\\Service};
\node[service] (payment) at (6,0.5) {Payment\\Service};

% Databases
\node[database] (auth-db) at (9,5) {Auth\\DB};
\node[database] (user-db) at (9,3.5) {User\\DB};
\node[database] (order-db) at (9,2) {Order\\DB};
\node[database] (payment-db) at (9,0.5) {Payment\\DB};

% Message Queue
\node[service, fill=orange!20] (queue) at (3,0.5) {Message\\Queue};

% Monitoring
\node[service, fill=red!20] (monitor) at (0,0.5) {Monitoring\\Logging};

% Connections
\draw[->] (mobile) -- (gateway);
\draw[->] (web) -- (gateway);
\draw[->] (gateway) -- (auth);
\draw[->] (gateway) -- (user-svc);
\draw[->] (gateway) -- (order);
\draw[->] (gateway) -- (payment);

\draw[->] (auth) -- (auth-db);
\draw[->] (user-svc) -- (user-db);
\draw[->] (order) -- (order-db);
\draw[->] (payment) -- (payment-db);

\draw[->] (order) -- (queue);
\draw[->] (payment) -- (queue);
\draw[->] (queue) -- (monitor);

% Service mesh connections (dashed)
\draw[dashed, gray] (auth) -- (user-svc);
\draw[dashed, gray] (user-svc) -- (order);
\draw[dashed, gray] (order) -- (payment);

\end{tikzpicture}
\caption{Arquitectura de microservicios con API Gateway y service mesh}
\end{figure}

\subsection{Patrones de Despliegue}
\begin{itemize}
    \item \textbf{Blue-Green Deployment:} Dos entornos idénticos para despliegues sin downtime
    \item \textbf{Canary Releases:} Despliegue gradual a un subconjunto de usuarios
    \item \textbf{Rolling Updates:} Actualización progresiva de instancias
    \item \textbf{Feature Flags:} Control dinámico de funcionalidades en producción
\end{itemize}

\subsection{Arquitectura de Pipeline CI/CD}
Un pipeline típico incluye las siguientes etapas:
\begin{enumerate}
    \item \textbf{Source Control:} Git con branching strategies (GitFlow, GitHub Flow)
    \item \textbf{Build:} Compilación y empaquetado automatizado
    \item \textbf{Test:} Pruebas unitarias, integración y end-to-end
    \item \textbf{Security Scan:} Análisis de vulnerabilidades y compliance
    \item \textbf{Deploy:} Despliegue automatizado a diferentes entornos
    \item \textbf{Monitor:} Observabilidad y alertas en tiempo real
\end{enumerate}

\newpage
\subsection{Diagrama: Pipeline CI/CD Detallado}
\begin{figure}[h!]
\centering
\begin{tikzpicture}[
    node distance=1.5cm,
    stage/.style={rectangle, draw, fill=blue!15, text width=2.2cm, text centered, minimum height=1cm, font=\footnotesize},
    arrow/.style={->, thick, blue},
    decision/.style={diamond, draw, fill=yellow!20, text width=2cm, text centered, font=\footnotesize, minimum height=1cm}
]

% Top row - CI Pipeline
\node[stage, fill=green!20] (source) {Source\\Control\\(Git)};
\node[stage, right=of source] (build) {Build \&\\Compile};
\node[stage, right=of build] (test) {Unit\\Tests};
\node[stage, right=of test] (security) {Security\\Scan};

% Bottom row - CD Pipeline
\node[stage, below=2.5cm of source, fill=orange!20] (integration) {Integration\\Tests};
\node[stage, right=of integration] (package) {Package\\Artifact};
\node[decision, right=of package] (approve) {Manual\\Approval};
\node[stage, right=of approve, fill=red!20] (deploy) {Deploy to\\Production};

% Vertical connection
\draw[arrow] (security) -- ++(0,-1.2cm) -| (integration);

% Horizontal arrows
\draw[arrow] (source) -- (build);
\draw[arrow] (build) -- (test);
\draw[arrow] (test) -- (security);
\draw[arrow] (integration) -- (package);
\draw[arrow] (package) -- (approve);
\draw[arrow] (approve) -- (deploy);

% Feedback loop
\draw[arrow, dashed, red, thick] (deploy) to[bend right=45] node[midway, above, font=\tiny] {Feedback} (source);

% Phase labels
\node[above=1.5cm of build, font=\Large\bfseries] {Pipeline CI/CD Completo};
\node[below=0.8cm of integration, font=\small, text=orange!70!black] {Continuous Integration};
\node[below=0.8cm of deploy, font=\small, text=red!70!black] {Continuous Deployment};

% Stage indicators
\node[left=0.5cm of source, font=\small, text=green!70!black] {CI};
\node[left=0.5cm of integration, font=\small, text=orange!70!black] {CD};

\end{tikzpicture}
\caption{Pipeline CI/CD: Flujo automatizado desde código hasta producción con retroalimentación}
\end{figure}

\subsection{Jenkins: Jenkinsfile Completo}
Pipeline declarativo con todas las etapas DevOps:
\begin{lstlisting}[language=groovy, caption=Jenkins - Pipeline completo, basicstyle=\ttfamily\tiny]
pipeline {
    agent {
        kubernetes {
            yaml """
                apiVersion: v1
                kind: Pod
                spec:
                  containers:
                  - name: node
                    image: node:18-alpine
                    command: ['sleep']
                    args: ['99d']
                  - name: docker
                    image: docker:dind
                    securityContext:
                      privileged: true
            """
        }
    }
    
    environment {
        DOCKER_REGISTRY = 'your-registry.com'
        IMAGE_NAME = 'devops-app'
        KUBECONFIG = credentials('kubeconfig')
        SONAR_TOKEN = credentials('sonar-token')
        SLACK_WEBHOOK = credentials('slack-webhook')
    }
    
    stages {
        stage('Checkout') {
            steps {
                checkout scm
                script {
                    env.GIT_COMMIT_SHORT = sh(
                        script: "git rev-parse --short HEAD",
                        returnStdout: true
                    ).trim()
                    env.BUILD_VERSION = "${env.BUILD_NUMBER}-${env.GIT_COMMIT_SHORT}"
                }
            }
        }
        
        stage('Code Quality') {
            parallel {
                stage('Lint') {
                    steps {
                        container('node') {
                            sh 'npm run lint'
                        }
                    }
                }
                stage('Security Audit') {
                    steps {
                        container('node') {
                            sh 'npm audit --audit-level high'
                        }
                    }
                }
            }
        }
        
        stage('Build & Deploy') {
            when { branch 'main' }
            steps {
                container('docker') {
                    sh '''
                        docker build -t ${DOCKER_REGISTRY}/${IMAGE_NAME}:${BUILD_VERSION} .
                        docker push ${DOCKER_REGISTRY}/${IMAGE_NAME}:${BUILD_VERSION}
                    '''
                }
            }
        }
    }
    
    post {
        success {
            sh 'echo "✅ Deployment successful"'
        }
        failure {
            sh 'echo "❌ Build failed"'
        }
    }
}
\end{lstlisting}

\section{Herramientas y Tecnologías DevOps}

\subsection{Control de Versiones y Colaboración}
\begin{itemize}
    \item \textbf{Git:} Sistema de control de versiones distribuido
    \item \textbf{GitHub/GitLab/Bitbucket:} Plataformas de hosting con CI/CD integrado
    \item \textbf{Branching Strategies:} GitFlow, GitHub Flow, GitLab Flow
\end{itemize}

\subsection{Integración y Entrega Continua (CI/CD)}
\begin{itemize}
    \item \textbf{Jenkins\index{Jenkins}:} Servidor de automatización open-source con plugins extensivos
    \item \textbf{GitLab CI\index{GitLab CI}:} CI/CD nativo integrado con GitLab
    \item \textbf{GitHub Actions\index{GitHub Actions}:} Workflows automatizados en GitHub
    \item \textbf{Azure DevOps\index{Azure DevOps}:} Suite completa de Microsoft para DevOps
    \item \textbf{CircleCI/Travis CI\index{CircleCI}\index{Travis CI}:} Servicios cloud de CI/CD
\end{itemize}

\subsection{Containerización y Orquestación}
\begin{itemize}
    \item \textbf{Docker:} Plataforma de containerización líder
    \item \textbf{Kubernetes:} Orquestador de contenedores de facto
    \item \textbf{OpenShift:} Plataforma empresarial basada en Kubernetes
    \item \textbf{Docker Swarm:} Orquestación nativa de Docker
    \item \textbf{Helm:} Gestor de paquetes para Kubernetes
\end{itemize}

\subsection{Infrastructure as Code (IaC)}
\begin{itemize}
    \item \textbf{Terraform\index{Terraform}:} Herramienta de aprovisionamiento multi-cloud
    \item \textbf{Ansible\index{Ansible}:} Automatización de configuración y despliegue
    \item \textbf{Puppet\index{Puppet}/Chef\index{Chef}:} Gestión de configuración empresarial
    \item \textbf{CloudFormation\index{CloudFormation}:} IaC nativo de AWS
    \item \textbf{ARM Templates\index{ARM Templates}:} Plantillas de Azure Resource Manager
\end{itemize}

\subsection{Monitoreo y Observabilidad}
\begin{itemize}
    \item \textbf{Prometheus\index{Prometheus} + Grafana\index{Grafana}:} Stack de monitoreo y visualización
    \item \textbf{ELK Stack\index{ELK Stack}:} Elasticsearch, Logstash, Kibana para logging
    \item \textbf{Jaeger\index{Jaeger}/Zipkin\index{Zipkin}:} Distributed tracing
    \item \textbf{New Relic\index{New Relic}/Datadog\index{Datadog}:} APM (Application Performance Monitoring)
    \item \textbf{Nagios\index{Nagios}/Zabbix\index{Zabbix}:} Monitoreo de infraestructura tradicional
\end{itemize}

\section{Containerización con Docker y Kubernetes}

\subsection{Docker: Fundamentos}
Docker revolucionó el desarrollo al proporcionar:
\begin{itemize}
    \item \textbf{Portabilidad:} "Funciona en mi máquina" se convierte en realidad
    \item \textbf{Consistencia:} Mismo entorno en desarrollo, testing y producción
    \item \textbf{Eficiencia:} Menor overhead comparado con VMs tradicionales
    \item \textbf{Escalabilidad:} Fácil replicación y distribución de aplicaciones
\end{itemize}

\subsection{Dockerfile: Ejemplo Práctico}
Dockerfile multi-stage para aplicación Node.js:
\begin{verbatim}
# Build stage
FROM node:18-alpine AS builder
WORKDIR /app
COPY package*.json ./
RUN npm ci --only=production && npm cache clean --force

# Development stage
FROM node:18-alpine AS development
WORKDIR /app
COPY package*.json ./
RUN npm install
COPY . .
EXPOSE 3000
CMD ["npm", "run", "dev"]

# Production stage
FROM node:18-alpine AS production
RUN addgroup -g 1001 -S nodejs
RUN adduser -S nextjs -u 1001
WORKDIR /app
COPY --from=builder /app/node_modules ./node_modules
COPY --chown=nextjs:nodejs . .
USER nextjs
EXPOSE 3000
ENV NODE_ENV=production
HEALTHCHECK --interval=30s --timeout=3s --start-period=5s --retries=3 \
  CMD curl -f http://localhost:3000/health || exit 1
CMD ["npm", "start"]
\end{verbatim}

\subsection{Docker Compose: Orquestación Local}
Configuración completa para desarrollo:
\begin{verbatim}
version: '3.8'
services:
  app:
    build:
      context: .
      target: development
    ports:
      - "3000:3000"
    volumes:
      - .:/app
      - /app/node_modules
    environment:
      - NODE_ENV=development
      - DATABASE_URL=postgresql://user:pass@db:5432/devops_db
      - REDIS_URL=redis://redis:6379
    depends_on:
      - db
      - redis
    networks:
      - devops-network
    restart: unless-stopped

  db:
    image: postgres:15-alpine
    environment:
      POSTGRES_DB: devops_db
      POSTGRES_USER: user
      POSTGRES_PASSWORD: pass
    volumes:
      - postgres_data:/var/lib/postgresql/data
      - ./init.sql:/docker-entrypoint-initdb.d/init.sql
    ports:
      - "5432:5432"
    networks:
      - devops-network
    restart: unless-stopped

  redis:
    image: redis:7-alpine
    ports:
      - "6379:6379"
    volumes:
      - redis_data:/data
    networks:
      - devops-network
    restart: unless-stopped

  nginx:
    image: nginx:alpine
    ports:
      - "80:80"
      - "443:443"
    volumes:
      - ./nginx.conf:/etc/nginx/nginx.conf
      - ./ssl:/etc/nginx/ssl
    depends_on:
      - app
    networks:
      - devops-network
    restart: unless-stopped

volumes:
  postgres_data:
  redis_data:

networks:
  devops-network:
    driver: bridge
\end{verbatim}

\subsection{Kubernetes: Orquestación Avanzada}
Kubernetes proporciona capacidades empresariales:
\begin{itemize}
    \item \textbf{Auto-scaling:} Escalado automático basado en métricas
    \item \textbf{Self-healing:} Recuperación automática de fallos
    \item \textbf{Service Discovery:} Descubrimiento automático de servicios
    \item \textbf{Rolling Updates:} Actualizaciones sin downtime
    \item \textbf{Resource Management:} Gestión eficiente de CPU y memoria
\end{itemize}

\subsection{Kubernetes: Manifiestos YAML}
Configuración completa de deployment y servicios:
\begin{verbatim}
# Namespace
apiVersion: v1
kind: Namespace
metadata:
  name: devops-app
---
# ConfigMap
apiVersion: v1
kind: ConfigMap
metadata:
  name: app-config
  namespace: devops-app
data:
  NODE_ENV: "production"
  PORT: "3000"
  DATABASE_HOST: "postgres-service"
  REDIS_HOST: "redis-service"
---
# Secret
apiVersion: v1
kind: Secret
metadata:
  name: app-secrets
  namespace: devops-app
type: Opaque
data:
  DATABASE_PASSWORD: cGFzc3dvcmQxMjM=  # password123 base64
  JWT_SECRET: c3VwZXJzZWNyZXRrZXk=      # supersecretkey base64
---
# Deployment
apiVersion: apps/v1
kind: Deployment
metadata:
  name: devops-app
  namespace: devops-app
  labels:
    app: devops-app
spec:
  replicas: 3
  strategy:
    type: RollingUpdate
    rollingUpdate:
      maxSurge: 1
      maxUnavailable: 1
  selector:
    matchLabels:
      app: devops-app
  template:
    metadata:
      labels:
        app: devops-app
    spec:
      containers:
      - name: app
        image: devops-app:latest
        ports:
        - containerPort: 3000
        envFrom:
        - configMapRef:
            name: app-config
        - secretRef:
            name: app-secrets
        resources:
          requests:
            memory: "128Mi"
            cpu: "100m"
          limits:
            memory: "512Mi"
            cpu: "500m"
        livenessProbe:
          httpGet:
            path: /health
            port: 3000
          initialDelaySeconds: 30
          periodSeconds: 10
        readinessProbe:
          httpGet:
            path: /ready
            port: 3000
          initialDelaySeconds: 5
          periodSeconds: 5
---
# Service
apiVersion: v1
kind: Service
metadata:
  name: devops-app-service
  namespace: devops-app
spec:
  selector:
    app: devops-app
  ports:
  - protocol: TCP
    port: 80
    targetPort: 3000
  type: ClusterIP
---
# Ingress
apiVersion: networking.k8s.io/v1
kind: Ingress
metadata:
  name: devops-app-ingress
  namespace: devops-app
  annotations:
    kubernetes.io/ingress.class: nginx
    cert-manager.io/cluster-issuer: letsencrypt-prod
    nginx.ingress.kubernetes.io/rate-limit: "100"
spec:
  tls:
  - hosts:
    - devops-app.example.com
    secretName: devops-app-tls
  rules:
  - host: devops-app.example.com
    http:
      paths:
      - path: /
        pathType: Prefix
        backend:
          service:
            name: devops-app-service
            port:
              number: 80
---
# HorizontalPodAutoscaler
apiVersion: autoscaling/v2
kind: HorizontalPodAutoscaler
metadata:
  name: devops-app-hpa
  namespace: devops-app
spec:
  scaleTargetRef:
    apiVersion: apps/v1
    kind: Deployment
    name: devops-app
  minReplicas: 3
  maxReplicas: 10
  metrics:
  - type: Resource
    resource:
      name: cpu
      target:
        type: Utilization
        averageUtilization: 70
  - type: Resource
    resource:
      name: memory
      target:
        type: Utilization
        averageUtilization: 80
\end{verbatim}

\section{Infrastructure as Code (IaC)}

\subsection{Principios de IaC}
\begin{itemize}
    \item \textbf{Versionado:} Infraestructura como código fuente
    \item \textbf{Reproducibilidad:} Entornos idénticos y predecibles
    \item \textbf{Automatización:} Eliminación de configuración manual
    \item \textbf{Documentación:} Código como documentación viviente
\end{itemize}

\subsection{Terraform: Ejemplo Completo}
Ejemplo completo de infraestructura web en AWS:
\begin{lstlisting}[language=bash, caption=Terraform - Infraestructura AWS completa]
# Provider configuration
terraform {
  required_providers {
    aws = {
      source  = "hashicorp/aws"
      version = "~> 5.0"
    }
  }
  backend "s3" {
    bucket = "devops-terraform-state"
    key    = "infrastructure/terraform.tfstate"
    region = "us-west-2"
  }
}

# VPC and networking
resource "aws_vpc" "main" {
  cidr_block           = "10.0.0.0/16"
  enable_dns_hostnames = true
  enable_dns_support   = true
  
  tags = {
    Name = "DevOps-VPC"
    Environment = "Production"
  }
}

resource "aws_subnet" "public" {
  count      = 2
  vpc_id     = aws_vpc.main.id
  cidr_block = "10.0.${count.index + 1}.0/24"
  availability_zone = data.aws_availability_zones.available.names[count.index]
  
  map_public_ip_on_launch = true
  
  tags = {
    Name = "Public-Subnet-${count.index + 1}"
    Type = "Public"
  }
}

# Application Load Balancer
resource "aws_lb" "main" {
  name               = "devops-alb"
  internal           = false
  load_balancer_type = "application"
  security_groups    = [aws_security_group.alb.id]
  subnets           = aws_subnet.public[*].id
  
  enable_deletion_protection = false
  
  tags = {
    Environment = "Production"
  }
}

# Auto Scaling Group
resource "aws_autoscaling_group" "web" {
  name                = "devops-asg"
  vpc_zone_identifier = aws_subnet.public[*].id
  target_group_arns   = [aws_lb_target_group.web.arn]
  health_check_type   = "ELB"
  
  min_size         = 2
  max_size         = 10
  desired_capacity = 3
  
  launch_template {
    id      = aws_launch_template.web.id
    version = "$Latest"
  }
  
  tag {
    key                 = "Name"
    value               = "DevOps-WebServer"
    propagate_at_launch = true
  }
}

# RDS Database
resource "aws_db_instance" "main" {
  identifier = "devops-database"
  
  engine         = "postgres"
  engine_version = "15.3"
  instance_class = "db.t3.micro"
  
  allocated_storage     = 20
  max_allocated_storage = 100
  storage_encrypted     = true
  
  db_name  = "devopsapp"
  username = "dbadmin"
  password = var.db_password
  
  vpc_security_group_ids = [aws_security_group.rds.id]
  db_subnet_group_name   = aws_db_subnet_group.main.name
  
  backup_retention_period = 7
  backup_window          = "03:00-04:00"
  maintenance_window     = "sun:04:00-sun:05:00"
  
  skip_final_snapshot = true
  
  tags = {
    Name = "DevOps-Database"
    Environment = "Production"
  }
}
\end{lstlisting}

\subsection{Ansible: Configuración de Servidores}
Playbook completo para configurar servidores web:
\begin{lstlisting}[language=yaml, caption=Ansible - Playbook de configuración]
---
- name: Configure Web Servers
  hosts: webservers
  become: yes
  vars:
    app_name: "devops-app"
    app_port: 3000
    nginx_port: 80
    
  tasks:
    - name: Update system packages
      apt:
        update_cache: yes
        upgrade: dist
        
    - name: Install required packages
      apt:
        name:
          - nginx
          - nodejs
          - npm
          - docker.io
          - docker-compose
          - git
          - htop
          - curl
        state: present
        
    - name: Start and enable Docker
      systemd:
        name: docker
        state: started
        enabled: yes
        
    - name: Create application directory
      file:
        path: "/opt/{{ app_name }}"
        state: directory
        owner: "{{ ansible_user }}"
        group: "{{ ansible_user }}"
        mode: '0755'
        
    - name: Clone application repository
      git:
        repo: "https://github.com/company/{{ app_name }}.git"
        dest: "/opt/{{ app_name }}"
        version: main
        force: yes
      become_user: "{{ ansible_user }}"
      
    - name: Configure Nginx
      template:
        src: nginx.conf.j2
        dest: "/etc/nginx/sites-available/{{ app_name }}"
        mode: '0644'
      notify: restart nginx
        
  handlers:
    - name: restart nginx
      systemd:
        name: nginx
        state: restarted
\end{lstlisting}

\section{Monitoreo y Observabilidad}

\subsection{Los Tres Pilares de la Observabilidad}
\begin{itemize}
    \item \textbf{Métricas:} Datos numéricos agregados sobre el comportamiento del sistema
    \item \textbf{Logs:} Registros detallados de eventos y transacciones
    \item \textbf{Traces:} Seguimiento de requests a través de sistemas distribuidos
\end{itemize}

\subsection{Implementación de Monitoreo}
\begin{itemize}
    \item \textbf{Golden Signals:} Latencia, tráfico, errores, saturación
    \item \textbf{SLI/SLO/SLA:} Service Level Indicators, Objectives, Agreements
    \item \textbf{Alerting:} Notificaciones proactivas basadas en umbrales
    \item \textbf{Dashboards:} Visualización en tiempo real del estado del sistema
\end{itemize}

\subsection{Herramientas de Monitoreo Específicas}
\begin{table}[h!]
\centering
\begin{tabular}{|l|l|l|}
\hline
\textbf{Categoría} & \textbf{Herramienta} & \textbf{Uso Principal} \\
\hline
Métricas & Prometheus & Recolección y almacenamiento \\
\hline
Visualización & Grafana & Dashboards y alertas \\
\hline
Logging & ELK Stack & Agregación y análisis de logs \\
\hline
APM & New Relic & Monitoreo de aplicaciones \\
\hline
Tracing & Jaeger & Distributed tracing \\
\hline
\end{tabular}
\caption{Herramientas de monitoreo por categoría}
\end{table}

\subsection{Prometheus: Configuración Completa}
Configuración de Prometheus con alertas y service discovery:
\begin{verbatim}
# prometheus.yml
global:
  scrape_interval: 15s
  evaluation_interval: 15s
  external_labels:
    cluster: 'production'
    region: 'us-west-2'

rule_files:
  - "alert_rules.yml"
  - "recording_rules.yml"

alerting:
  alertmanagers:
    - static_configs:
        - targets:
          - alertmanager:9093

scrape_configs:
  # Prometheus itself
  - job_name: 'prometheus'
    static_configs:
      - targets: ['localhost:9090']

  # Node Exporter
  - job_name: 'node-exporter'
    kubernetes_sd_configs:
      - role: endpoints
        namespaces:
          names:
            - monitoring
    relabel_configs:
      - source_labels: [__meta_kubernetes_service_name]
        action: keep
        regex: node-exporter

  # Application metrics
  - job_name: 'devops-app'
    kubernetes_sd_configs:
      - role: pod
        namespaces:
          names:
            - production
            - staging
    relabel_configs:
      - source_labels: [__meta_kubernetes_pod_label_app]
        action: keep
        regex: devops-app
      - source_labels: [__meta_kubernetes_pod_annotation_prometheus_io_scrape]
        action: keep
        regex: true
      - source_labels: [__meta_kubernetes_pod_annotation_prometheus_io_path]
        action: replace
        target_label: __metrics_path__
        regex: (.+)

  # Database metrics
  - job_name: 'postgres-exporter'
    static_configs:
      - targets: ['postgres-exporter:9187']
    scrape_interval: 30s

  # Redis metrics
  - job_name: 'redis-exporter'
    static_configs:
      - targets: ['redis-exporter:9121']
\end{verbatim}

\subsection{Alertmanager: Reglas de Alertas}
Configuración de alertas críticas para producción:
\begin{verbatim}
# alert_rules.yml
groups:
- name: application.rules
  rules:
  - alert: HighErrorRate
    expr: |
      (
        rate(http_requests_total{status=~"5.."}[5m]) /
        rate(http_requests_total[5m])
      ) > 0.05
    for: 5m
    labels:
      severity: critical
      service: devops-app
    annotations:
      summary: "High error rate detected"
      description: "Error rate is {{ $value | humanizePercentage }} for {{ $labels.instance }}"

  - alert: HighLatency
    expr: |
      histogram_quantile(0.95, 
        rate(http_request_duration_seconds_bucket[5m])
      ) > 0.5
    for: 2m
    labels:
      severity: warning
      service: devops-app
    annotations:
      summary: "High latency detected"
      description: "95th percentile latency is {{ $value }}s for {{ $labels.instance }}"

  - alert: PodCrashLooping
    expr: |
      rate(kube_pod_container_status_restarts_total[15m]) > 0
    for: 5m
    labels:
      severity: critical
    annotations:
      summary: "Pod is crash looping"
      description: "Pod {{ $labels.pod }} in namespace {{ $labels.namespace }} is restarting frequently"

- name: infrastructure.rules
  rules:
  - alert: HighCPUUsage
    expr: |
      (
        100 - (avg by(instance) (irate(node_cpu_seconds_total{mode="idle"}[5m])) * 100)
      ) > 80
    for: 5m
    labels:
      severity: warning
    annotations:
      summary: "High CPU usage"
      description: "CPU usage is {{ $value }}% on {{ $labels.instance }}"

  - alert: HighMemoryUsage
    expr: |
      (
        (node_memory_MemTotal_bytes - node_memory_MemAvailable_bytes) / 
        node_memory_MemTotal_bytes
      ) > 0.85
    for: 5m
    labels:
      severity: warning
    annotations:
      summary: "High memory usage"
      description: "Memory usage is {{ $value | humanizePercentage }} on {{ $labels.instance }}"

  - alert: DiskSpaceLow
    expr: |
      (
        (node_filesystem_size_bytes - node_filesystem_free_bytes) /
        node_filesystem_size_bytes
      ) > 0.85
    for: 5m
    labels:
      severity: critical
    annotations:
      summary: "Disk space low"
      description: "Disk usage is {{ $value | humanizePercentage }} on {{ $labels.instance }}"
\end{verbatim}

\subsection{Grafana: Dashboard as Code}
Configuración de dashboard en JSON para importación automática:
\begin{verbatim}
{
  "dashboard": {
    "id": null,
    "title": "DevOps Application Dashboard",
    "tags": ["devops", "application"],
    "timezone": "browser",
    "panels": [
      {
        "id": 1,
        "title": "Request Rate",
        "type": "graph",
        "targets": [
          {
            "expr": "rate(http_requests_total[5m])",
            "legendFormat": "{{ instance }} - {{ method }}"
          }
        ],
        "yAxes": [
          {
            "label": "Requests/sec",
            "min": 0
          }
        ]
      },
      {
        "id": 2,
        "title": "Error Rate",
        "type": "singlestat",
        "targets": [
          {
            "expr": "rate(http_requests_total{status=~\"5..\"}[5m]) / rate(http_requests_total[5m])",
            "legendFormat": "Error Rate"
          }
        ],
        "valueMaps": [
          {
            "value": "null",
            "text": "0%"
          }
        ],
        "thresholds": "0.01,0.05",
        "colorBackground": true
      },
      {
        "id": 3,
        "title": "Response Time",
        "type": "graph",
        "targets": [
          {
            "expr": "histogram_quantile(0.50, rate(http_request_duration_seconds_bucket[5m]))",
            "legendFormat": "50th percentile"
          },
          {
            "expr": "histogram_quantile(0.95, rate(http_request_duration_seconds_bucket[5m]))",
            "legendFormat": "95th percentile"
          },
          {
            "expr": "histogram_quantile(0.99, rate(http_request_duration_seconds_bucket[5m]))",
            "legendFormat": "99th percentile"
          }
        ]
      }
    ],
    "time": {
      "from": "now-1h",
      "to": "now"
    },
    "refresh": "30s"
  }
}
\end{verbatim}

\section{DevSecOps: Seguridad Integrada}

\subsection{Principios de DevSecOps}
DevSecOps integra la seguridad en todo el ciclo de vida:
\begin{itemize}
    \item \textbf{Shift Left:} Incorporar seguridad desde las primeras fases
    \item \textbf{Automatización:} Escaneos automáticos de vulnerabilidades
    \item \textbf{Compliance as Code:} Políticas de seguridad como código
    \item \textbf{Zero Trust:} Verificación continua de identidad y acceso
\end{itemize}

\subsection{Herramientas de Seguridad}
\begin{itemize}
    \item \textbf{SAST:} Static Application Security Testing (SonarQube, Checkmarx)
    \item \textbf{DAST:} Dynamic Application Security Testing (OWASP ZAP)
    \item \textbf{Container Security:} Twistlock, Aqua Security
    \item \textbf{Secrets Management:} HashiCorp Vault, AWS Secrets Manager
    \item \textbf{Compliance:} Chef InSpec, Open Policy Agent
\end{itemize}

\subsection{Pipeline de Seguridad}
Un pipeline DevSecOps típico incluye:
\begin{enumerate}
    \item \textbf{Pre-commit hooks:} Validación de secretos y políticas
    \item \textbf{SAST scanning:} Análisis estático del código
    \item \textbf{Dependency scanning:} Verificación de vulnerabilidades en librerías
    \item \textbf{Container scanning:} Análisis de imágenes Docker
    \item \textbf{DAST testing:} Pruebas dinámicas en entornos de staging
    \item \textbf{Runtime protection:} Monitoreo de seguridad en producción
\end{enumerate}

\section{Metodologías Ágiles y DevOps}

\subsection{Integración con Scrum}
DevOps complementa Scrum proporcionando:
\begin{itemize}
    \item \textbf{Definition of Done:} Incluye criterios de despliegue y monitoreo
    \item \textbf{Sprint Reviews:} Demostraciones en entornos productivos
    \item \textbf{Retrospectivas:} Mejora continua de procesos DevOps
    \item \textbf{Velocity:} Métricas de entrega real a producción
\end{itemize}

\subsection{Lean Software Development}
Principios Lean aplicados a DevOps:
\begin{itemize}
    \item \textbf{Eliminar desperdicios:} Automatizar tareas repetitivas
    \item \textbf{Amplificar el aprendizaje:} Feedback loops rápidos
    \item \textbf{Decidir lo más tarde posible:} Feature flags y A/B testing
    \item \textbf{Entregar rápido:} Continuous deployment
\end{itemize}

\section{Beneficios de DevOps}
Implementar DevOps trae múltiples ventajas cuantificables:

\subsection{Beneficios Técnicos}
\begin{itemize}
    \item \textbf{Reducción de errores:} 50-90\% menos errores de despliegue
    \item \textbf{Tiempo de recuperación:} MTTR reducido de horas a minutos
    \item \textbf{Frecuencia de despliegue:} De mensual a múltiples veces por día
    \item \textbf{Lead time:} Reducción del 50-90\% en tiempo de entrega
\end{itemize}

\subsection{Beneficios de Negocio}
\begin{itemize}
    \item \textbf{Time-to-market:} Lanzamiento más rápido de features
    \item \textbf{Satisfacción del cliente:} Mayor calidad y disponibilidad
    \item \textbf{Productividad del equipo:} Menos tiempo en tareas manuales
    \item \textbf{Competitividad:} Capacidad de adaptación rápida al mercado
\end{itemize}

\section{Casos de Estudio Detallados}

\subsection{Netflix: Transformación a Escala Global}
Netflix es el ejemplo más citado de transformación DevOps exitosa:

\subsubsection{Métricas de Impacto}
\begin{itemize}
    \item \textbf{Despliegues:} 4,000+ despliegues por día
    \item \textbf{Microservicios:} 700+ servicios en producción
    \item \textbf{Disponibilidad:} 99.99\% uptime global
    \item \textbf{Tiempo de recuperación:} < 5 minutos MTTR
    \item \textbf{Escalabilidad:} 200+ millones de usuarios globales
\end{itemize}

\subsubsection{Arquitectura y Herramientas}
\begin{itemize}
    \item \textbf{Cloud Native:} 100\% AWS desde 2016
    \item \textbf{Chaos Engineering:} Chaos Monkey y Simian Army
    \item \textbf{CI/CD:} Spinnaker (open source)
    \item \textbf{Monitoreo:} Atlas (métricas internas)
    \item \textbf{A/B Testing:} Experimentación continua en producción
\end{itemize}

\subsubsection{Lecciones Aprendidas}
\begin{itemize}
    \item \textbf{Cultura primero:} Freedom and responsibility culture
    \item \textbf{Automatización total:} Eliminación de procesos manuales
    \item \textbf{Fallo como aprendizaje:} Chaos engineering proactivo
    \item \textbf{Datos como guía:} Decisiones basadas en métricas
\end{itemize}

\subsection{Amazon: Pioneros en DevOps}
Amazon estableció muchas de las prácticas DevOps modernas:

\subsubsection{Métricas de Rendimiento}
\begin{itemize}
    \item \textbf{Frecuencia de despliegue:} Cada 11.7 segundos
    \item \textbf{Lead time:} < 60 minutos desde commit a producción
    \item \textbf{MTTR:} < 60 segundos para rollback automático
    \item \textbf{Change failure rate:} < 0.1\%
    \item \textbf{Equipos:} 2-pizza teams (6-8 personas máximo)
\end{itemize}

\subsubsection{Innovaciones Técnicas}
\begin{itemize}
    \item \textbf{Immutable Infrastructure:} AMIs y contenedores
    \item \textbf{Blue-Green Deployments:} Despliegues sin downtime
    \item \textbf{Circuit Breakers:} Resiliencia ante fallos
    \item \textbf{Service Mesh:} Comunicación segura entre servicios
\end{itemize}

\subsection{Spotify: Modelo Organizacional}
Spotify creó un modelo organizacional único para DevOps:

\subsubsection{Estructura de Equipos}
\begin{itemize}
    \item \textbf{Squads:} Equipos autónomos de 6-12 personas
    \item \textbf{Tribes:} Colección de squads (< 100 personas)
    \item \textbf{Chapters:} Comunidades de práctica por especialidad
    \item \textbf{Guilds:} Comunidades de interés transversales
\end{itemize}

\subsubsection{Resultados Cuantificables}
\begin{itemize}
    \item \textbf{Despliegues:} 10,000+ por día
    \item \textbf{Lead time:} < 30 minutos promedio
    \item \textbf{Autonomía:} 99\% de decisiones tomadas por squads
    \item \textbf{Satisfacción:} 4.2/5 satisfacción de desarrolladores
\end{itemize}

\section{Casos de Éxito en México y Latinoamérica}

\subsection{Mercado Libre: Líder DevOps en Latinoamérica}
Mercado Libre es el caso de éxito más destacado de DevOps en la región:

\subsubsection{Transformación Digital}
\begin{itemize}
    \item \textbf{Escala:} 200+ millones de usuarios activos en 18 países
    \item \textbf{Transacciones:} 1.5+ billones de transacciones anuales
    \item \textbf{Infraestructura:} 100\% cloud-native en AWS desde 2019
    \item \textbf{Equipos:} 8,000+ desarrolladores distribuidos
\end{itemize}

\subsubsection{Métricas de Rendimiento}
\begin{itemize}
    \item \textbf{Despliegues:} 15,000+ despliegues por semana
    \item \textbf{Disponibilidad:} 99.9\% uptime durante eventos como Hot Sale
    \item \textbf{Lead time:} < 2 horas desde commit a producción
    \item \textbf{MTTR:} < 15 minutos para rollback automático
    \item \textbf{Microservicios:} 2,000+ servicios en producción
\end{itemize}

\subsubsection{Herramientas y Tecnologías}
\begin{itemize}
    \item \textbf{Orchestración:} Kubernetes con Istio service mesh
    \item \textbf{CI/CD:} Jenkins + GitLab CI con pipelines paralelos
    \item \textbf{Monitoreo:} Stack ELK + Prometheus + Grafana
    \item \textbf{Chaos Engineering:} Fury (herramienta propia)
    \item \textbf{Feature Flags:} Sistema propio para A/B testing
\end{itemize}

\subsection{Banco Azteca: DevOps en Servicios Financieros}
Transformación DevOps en el sector financiero mexicano:

\subsubsection{Desafíos del Sector}
\begin{itemize}
    \item \textbf{Regulación:} Cumplimiento CNBV y normativas financieras
    \item \textbf{Seguridad:} PCI-DSS y protección de datos sensibles
    \item \textbf{Legacy:} Integración con sistemas mainframe existentes
    \item \textbf{Disponibilidad:} 99.95\% SLA requerido por regulación
\end{itemize}

\subsubsection{Implementación Gradual}
\begin{itemize}
    \item \textbf{Fase 1:} Modernización de aplicaciones web (2020-2021)
    \item \textbf{Fase 2:} Containerización de servicios core (2021-2022)
    \item \textbf{Fase 3:} Implementación de CI/CD completo (2022-2023)
    \item \textbf{Fase 4:} Observabilidad y chaos engineering (2023-2024)
\end{itemize}

\subsubsection{Resultados Obtenidos}
\begin{itemize}
    \item \textbf{Despliegues:} De mensual a semanal (700\% mejora)
    \item \textbf{Incidentes:} Reducción del 60\% en incidentes críticos
    \item \textbf{Time-to-market:} Reducción del 40\% en nuevas funcionalidades
    \item \textbf{Compliance:} 100\% trazabilidad para auditorías
\end{itemize}

\subsection{Rappi: Startup DevOps-Native}
Caso de estudio de startup colombiana con adopción DevOps desde el inicio:

\subsubsection{Crecimiento Exponencial}
\begin{itemize}
    \item \textbf{Expansión:} De 1 a 9 países en 5 años
    \item \textbf{Usuarios:} 20+ millones de usuarios activos
    \item \textbf{Pedidos:} 500+ millones de pedidos procesados
    \item \textbf{Ciudades:} 250+ ciudades en Latinoamérica
\end{itemize}

\subsubsection{Arquitectura DevOps}
\begin{itemize}
    \item \textbf{Microservicios:} 500+ servicios independientes
    \item \textbf{Multi-cloud:} AWS + GCP para redundancia
    \item \textbf{Edge computing:} CDN distribuido por región
    \item \textbf{Real-time:} Procesamiento de eventos en tiempo real
\end{itemize}

\subsubsection{Métricas de Operación}
\begin{itemize}
    \item \textbf{Latencia:} < 200ms promedio en toda la región
    \item \textbf{Escalabilidad:} Auto-scaling para picos de demanda (3x)
    \item \textbf{Deployment:} 200+ despliegues por día
    \item \textbf{Recovery:} < 5 minutos MTTR con blue-green deployments
\end{itemize}

\section{DevOps en el Sector Público LATAM}

\subsection{Gobierno Digital México}
La transformación digital del gobierno mexicano representa un caso único de DevOps a escala nacional:

\subsubsection{Plataforma gob.mx}
\begin{itemize}
    \item \textbf{Escala:} 300+ trámites digitalizados
    \item \textbf{Usuarios:} 50+ millones de ciudadanos registrados
    \item \textbf{Disponibilidad:} 99.5\% SLA gubernamental
    \item \textbf{Arquitectura:} Microservicios en AWS GovCloud
\end{itemize}

\subsubsection{Desafíos Únicos del Sector Público}
\begin{itemize}
    \item \textbf{Regulación estricta:} Ley Federal de Transparencia y Acceso a la Información
    \item \textbf{Seguridad nacional:} Protección de datos ciudadanos sensibles
    \item \textbf{Interoperabilidad:} Integración entre múltiples dependencias
    \item \textbf{Presupuesto limitado:} Optimización de costos con herramientas open source
\end{itemize}

\subsection{Brasil: Serpro - Modernización Federal}
Servicio Federal de Procesamiento de Datos de Brasil:

\subsubsection{Transformación DevOps}
\begin{itemize}
    \item \textbf{Legacy modernization:} Migración de mainframes COBOL a microservicios
    \item \textbf{Cloud híbrida:} Combinación de on-premises y nube pública
    \item \textbf{APIs públicas:} 200+ APIs para desarrolladores externos
    \item \textbf{Impacto:} Reducción del 60\% en tiempo de desarrollo
\end{itemize}

\subsection{Startups Unicornio LATAM}

\subsubsection{Nubank: DevOps en Fintech}
El banco digital más grande de Latinoamérica:

\begin{itemize}
    \item \textbf{Escala técnica:} 70+ millones de usuarios activos
    \item \textbf{Arquitectura:} 1,000+ microservicios en Clojure/Scala
    \item \textbf{Despliegues:} 500+ despliegues por día
    \item \textbf{Infraestructura:} Multi-cloud (AWS + GCP) para resiliencia
    \item \textbf{Regulación:} Cumplimiento bancario en múltiples países
\end{itemize}

\subsubsection{Kavak: Marketplace Automotriz}
Expansión acelerada con DevOps:

\begin{itemize}
    \item \textbf{Crecimiento:} De 1 a 4 países en 3 años
    \item \textbf{Transacciones:} \$2B+ en vehículos vendidos
    \item \textbf{Tecnología:} Node.js + React con Kubernetes
    \item \textbf{CI/CD:} GitLab CI con despliegues automáticos
    \item \textbf{Monitoreo:} Datadog + New Relic para observabilidad
\end{itemize}

\subsubsection{Cornershop (Uber): Delivery a Escala}
Adquisición exitosa basada en arquitectura DevOps:

\begin{itemize}
    \item \textbf{Adquisición:} Comprada por Uber por \$3B
    \item \textbf{Integración:} Migración a plataforma Uber en 18 meses
    \item \textbf{Escalabilidad:} De 100K a 10M+ pedidos mensuales
    \item \textbf{Latencia:} < 200ms en toda LATAM
\end{itemize}

\subsection{Lecciones para el Contexto Latinoamericano}

\subsubsection{Desafíos Regionales}
\begin{itemize}
    \item \textbf{Talento:} Escasez de profesionales DevOps senior
    \item \textbf{Infraestructura:} Conectividad variable entre países
    \item \textbf{Regulación:} Marcos normativos heterogéneos
    \item \textbf{Presupuesto:} Limitaciones de inversión en herramientas
\end{itemize}

\subsubsection{Estrategias de Éxito}
\begin{itemize}
    \item \textbf{Capacitación local:} Programas de formación interna
    \item \textbf{Herramientas open source:} Reducir costos de licenciamiento
    \item \textbf{Comunidades:} DevOps México, Colombia DevOps, Brasil DevOps
    \item \textbf{Partnerships:} Colaboración con universidades locales
\end{itemize}

\subsubsection{Oportunidades de Crecimiento}
\begin{itemize}
    \item \textbf{Gobierno digital:} Modernización de servicios públicos
    \item \textbf{Fintech:} Inclusión financiera con tecnología
    \item \textbf{E-commerce:} Crecimiento del comercio electrónico
    \item \textbf{Healthtech:} Telemedicina y salud digital
\end{itemize}

\section{Implementación Práctica: Checklist DevOps}

\subsection{Fase 1: Fundamentos (Semanas 1-4)}
\begin{itemize}
    \item[$\square$] \textbf{Control de versiones:} Git con branching strategy definida
    \item[$\square$] \textbf{CI básico:} Build y tests automáticos en cada commit
    \item[$\square$] \textbf{Ambientes:} Separación dev/staging/production
    \item[$\square$] \textbf{Documentación:} README, arquitectura, runbooks
    \item[$\square$] \textbf{Métricas básicas:} Uptime, response time, error rate
\end{itemize}

\subsection{Fase 2: Automatización (Semanas 5-8)}
\begin{itemize}
    \item[$\square$] \textbf{CD Pipeline:} Despliegue automático a staging
    \item[$\square$] \textbf{Infrastructure as Code:} Terraform o CloudFormation
    \item[$\square$] \textbf{Containerización:} Docker para aplicaciones
    \item[$\square$] \textbf{Testing:} Unit, integration, y smoke tests
    \item[$\square$] \textbf{Security scanning:} SAST y dependency checks
\end{itemize}

\subsection{Fase 3: Observabilidad (Semanas 9-12)}
\begin{itemize}
    \item[$\square$] \textbf{Logging centralizado:} ELK Stack o equivalente
    \item[$\square$] \textbf{Métricas avanzadas:} Prometheus + Grafana
    \item[$\square$] \textbf{Alertas:} Alertmanager con escalación
    \item[$\square$] \textbf{Distributed tracing:} Jaeger o Zipkin
    \item[$\square$] \textbf{Dashboards:} Business y technical metrics
\end{itemize}

\subsection{Fase 4: Optimización (Semanas 13-16)}
\begin{itemize}
    \item[$\square$] \textbf{Auto-scaling:} HPA y VPA en Kubernetes
    \item[$\square$] \textbf{Chaos engineering:} Pruebas de resiliencia
    \item[$\square$] \textbf{Feature flags:} Despliegues graduales
    \item[$\square$] \textbf{A/B testing:} Experimentación en producción
    \item[$\square$] \textbf{Performance optimization:} Profiling y tuning
\end{itemize}

\subsection{Métricas de Éxito}
\begin{table}[h!]
\centering
\begin{tabular}{|l|l|l|l|}
\hline
\textbf{Métrica} & \textbf{Baseline} & \textbf{Objetivo 6m} & \textbf{Clase Mundial} \\
\hline
Deployment Frequency & Mensual & Semanal & Múltiple/día \\
\hline
Lead Time & 1-6 meses & < 1 semana & < 1 día \\
\hline
MTTR & 1-7 días & < 1 día & < 1 hora \\
\hline
Change Failure Rate & 31-45\% & 16-30\% & 0-15\% \\
\hline
\end{tabular}
\caption{Evolución esperada de métricas DORA}
\end{table}

\section{Estado de la Investigación DevOps}

\subsection{Papers Académicos Recientes}
Tendencias en la investigación DevOps 2023-2024:

\subsubsection{Journals de Alto Impacto}
\begin{itemize}
    \item \textbf{IEEE Software:} "AI-Driven DevOps: A Systematic Literature Review"
    \item \textbf{ACM Computing Surveys:} "Security in DevOps: A Comprehensive Analysis"
    \item \textbf{Journal of Systems and Software:} "Microservices DevOps Patterns"
    \item \textbf{Empirical Software Engineering:} "DORA Metrics Validation Study"
\end{itemize}

\subsection{Tendencias en Conferencias IEEE/ACM}
\begin{itemize}
    \item \textbf{ICSE 2024:} 15\% de papers relacionados con DevOps
    \item \textbf{ASE 2024:} Focus en automated testing y CI/CD
    \item \textbf{MSR 2024:} Mining software repositories para DevOps insights
    \item \textbf{ESEC/FSE 2024:} DevSecOps y compliance automation
\end{itemize}

\subsection{Proyectos Open Source Emergentes}
\begin{table}[h!]
\centering
\begin{tabular}{|l|l|l|l|}
\hline
\textbf{Proyecto} & \textbf{Categoría} & \textbf{Stars} & \textbf{Tendencia} \\
\hline
Backstage & Developer Portal & 25K+ & ↗️ Creciendo \\
\hline
Crossplane & Infrastructure & 8K+ & ↗️ Creciendo \\
\hline
Flux & GitOps & 6K+ & ↗️ Creciendo \\
\hline
Tekton & CI/CD Native & 8K+ & → Estable \\
\hline
OpenTelemetry & Observability & 3K+ & ↗️ Creciendo \\
\hline
\end{tabular}
\caption{Proyectos open source DevOps emergentes 2024}
\end{table}

\section{Troubleshooting y Post-Mortems LATAM}

Esta sección presenta casos reales de debugging y resolución de incidentes en empresas latinoamericanas, proporcionando playbooks prácticos para situaciones críticas.

\subsection{Caso 1: Caída Masiva Black Friday - Mercado Libre 2023}

\textbf{Contexto:} Durante el Black Friday 2023, Mercado Libre experimentó una caída parcial que afectó al 15\% de las transacciones durante 47 minutos.

\textbf{Síntomas Iniciales:}
\begin{itemize}
    \item Latencia de API incrementada de 200ms a 8000ms
    \item Error rate aumentó del 0.1\% al 12\%
    \item Alertas de memoria en servicios de checkout
    \item Queues de Redis saturadas
\end{itemize}

\textbf{Root Cause Identificado:}
\begin{itemize}
    \item Memory leak en servicio de recomendaciones
    \item Configuración incorrecta de JVM heap size
    \item Falta de circuit breakers en llamadas externas
    \item Redis sin configuración de eviction policy
\end{itemize}

\textbf{Comandos de Debugging Utilizados:}
\begin{lstlisting}[language=bash, caption=Investigación del incidente]
# Verificar estado de pods Kubernetes
kubectl get pods -n checkout --sort-by='.status.containerStatuses[0].restartCount'

# Analizar logs de aplicación
kubectl logs -f deployment/checkout-service --tail=1000 | grep ERROR

# Verificar métricas de memoria
kubectl top pods -n checkout --sort-by=memory

# Revisar configuración de Redis
redis-cli info memory
redis-cli slowlog get 10
\end{lstlisting}

\textbf{Lecciones Aprendidas:}
\begin{enumerate}
    \item Implementar circuit breakers obligatorios
    \item Configurar memory limits más conservadores
    \item Establecer alertas proactivas de memoria al 70\%
    \item Implementar chaos engineering regular
\end{enumerate}

\subsection{Caso 2: Incident de Seguridad - Nubank 2024}

\textbf{Contexto:} Detección de acceso no autorizado a logs conteniendo información sensible de clientes.

\textbf{Timeline del Incidente:}
\begin{itemize}
    \item \textbf{09:15} - Alerta de SIEM por patrones anómalos
    \item \textbf{09:22} - Confirmación de acceso no autorizado
    \item \textbf{09:30} - Activación de protocolo de seguridad
    \item \textbf{09:45} - Rotación de credenciales comprometidas
    \item \textbf{10:30} - Implementación de parches
    \item \textbf{12:00} - Notificación a autoridades regulatorias
\end{itemize}

\textbf{Comandos de Auditoría:}
\begin{lstlisting}[language=bash, caption=Investigación forense]
# Revisar logs de acceso ELK Stack
curl -X GET "elasticsearch:9200/logs-*/_search" -H 'Content-Type: application/json' -d'
{
  "query": {
    "range": {
      "@timestamp": {
        "gte": "2024-01-15T09:00:00",
        "lte": "2024-01-15T10:00:00"
      }
    }
  }
}'

# Analizar accesos SSH
sudo ausearch -ts 09:00 -te 10:00 -k access
sudo grep "Accepted publickey" /var/log/auth.log
\end{lstlisting}

\subsection{Playbook de Respuesta a Incidentes LATAM}

\textbf{Protocolo de Escalación:}
\begin{enumerate}
    \item \textbf{Severidad 1 (Crítico):} Respuesta en 15 minutos
    \item \textbf{Severidad 2 (Alto):} Respuesta en 1 hora
    \item \textbf{Severidad 3 (Medio):} Respuesta en 4 horas
    \item \textbf{Severidad 4 (Bajo):} Respuesta en 24 horas
\end{enumerate}

\textbf{Toolkit de Diagnóstico Rápido:}
\begin{lstlisting}[language=bash, caption=Script de diagnóstico esencial]
#!/bin/bash
# Diagnóstico rápido DevOps LATAM

echo "=== KUBERNETES HEALTH ==="
kubectl get nodes
kubectl get pods --all-namespaces | grep -v Running

echo "=== RESOURCE USAGE ==="
kubectl top nodes
kubectl top pods --all-namespaces --sort-by=memory

echo "=== DATABASE STATUS ==="
psql -c "SELECT count(*) as connections, state FROM pg_stat_activity GROUP BY state;"

echo "=== REDIS STATUS ==="
redis-cli info memory | grep used_memory_human

echo "=== RECENT ERRORS ==="
tail -100 /var/log/application.log | grep ERROR
\end{lstlisting}

\textbf{Checklist Post-Mortem:}
\begin{itemize}
    \item[$\square$] Timeline detallado del incidente
    \item[$\square$] Root cause analysis completo
    \item[$\square$] Impacto en usuarios cuantificado
    \item[$\square$] Acciones correctivas implementadas
    \item[$\square$] Medidas preventivas definidas
    \item[$\square$] Documentación actualizada
\end{itemize}

\section{Carrera Profesional en DevOps}

\subsection{Roadmap de Habilidades DevOps}
\begin{figure}[h!]
\centering
\begin{tikzpicture}[
    node distance=1.5cm,
    level/.style={rectangle, draw, fill=blue!20, text width=3cm, text centered, minimum height=1.2cm, font=\footnotesize},
    arrow/.style={->, thick, blue}
]

% Career levels
\node[level, fill=green!20] (junior) {Junior DevOps\\Engineer\\(0-2 años)};
\node[level, above=of junior] (mid) {DevOps\\Engineer\\(2-5 años)};
\node[level, above=of mid] (senior) {Senior DevOps\\Engineer\\(5-8 años)};
\node[level, above=of senior] (lead) {DevOps Team\\Lead\\(8-12 años)};
\node[level, above=of lead] (architect) {DevOps\\Architect\\(12+ años)};

% Skills for each level
\node[right=3cm of junior, text width=6cm, font=\tiny] (jskills) {
    \textbf{Skills:} Linux, Git, Docker, Jenkins básico, Scripting (Bash/Python), 
    Networking básico, Cloud fundamentals
};

\node[right=3cm of mid, text width=6cm, font=\tiny] (mskills) {
    \textbf{Skills:} Kubernetes, Terraform, Ansible, CI/CD avanzado, 
    Monitoring (Prometheus/Grafana), Security scanning, IaC
};

\node[right=3cm of senior, text width=6cm, font=\tiny] (sskills) {
    \textbf{Skills:} Arquitectura de sistemas, Multi-cloud, GitOps, 
    Chaos Engineering, Performance tuning, Mentoring
};

\node[right=3cm of lead, text width=6cm, font=\tiny] (lskills) {
    \textbf{Skills:} Team leadership, Strategy, Budget management, 
    Stakeholder communication, Process optimization
};

\node[right=3cm of architect, text width=6cm, font=\tiny] (askills) {
    \textbf{Skills:} Enterprise architecture, Technology roadmaps, 
    Cross-functional leadership, Innovation, Industry expertise
};

% Arrows
\draw[arrow] (junior) -- (mid);
\draw[arrow] (mid) -- (senior);
\draw[arrow] (senior) -- (lead);
\draw[arrow] (lead) -- (architect);

% Skill connections
\draw[dashed] (junior) -- (jskills);
\draw[dashed] (mid) -- (mskills);
\draw[dashed] (senior) -- (sskills);
\draw[dashed] (lead) -- (lskills);
\draw[dashed] (architect) -- (askills);

% Title
\node[above=1cm of architect, font=\Large\bfseries] {DevOps Career Progression};

\end{tikzpicture}
\caption{Roadmap de carrera profesional en DevOps}
\end{figure}

\subsection{Certificaciones Relevantes}
\begin{table}[h!]
\centering
\begin{tabular}{|l|l|l|l|}
\hline
\textbf{Certificación} & \textbf{Proveedor} & \textbf{Nivel} & \textbf{Costo USD} \\
\hline
AWS DevOps Engineer & Amazon & Professional & \$300 \\
\hline
Azure DevOps Expert & Microsoft & Expert & \$165 \\
\hline
GCP DevOps Engineer & Google & Professional & \$200 \\
\hline
CKA/CKAD & CNCF & Professional & \$395 \\
\hline
Docker Certified & Docker & Associate & \$195 \\
\hline
Terraform Associate & HashiCorp & Associate & \$70 \\
\hline
\end{tabular}
\caption{Certificaciones DevOps más valoradas en el mercado}
\end{table}

\subsection{Salarios por Región LATAM (2024)}
\begin{figure}[H]
\centering
\begin{tikzpicture}
\begin{axis}[
    title={Salarios DevOps por País (USD anuales)},
    xlabel={Nivel de Experiencia},
    ylabel={Salario Anual (USD)},
    ybar,
    bar width=12pt,
    width=15cm,
    height=9cm,
    ymin=0,
    ymax=130000,
    symbolic x coords={Junior,Mid,Senior,Lead,Architect},
    xtick=data,
    x tick label style={font=\small},
    legend pos=north west,
    legend style={font=\footnotesize},
    legend entries={México,Brasil,Argentina,Colombia,Chile},
    enlarge x limits=0.1
]

% México
\addplot[fill=green!70, draw=green!90, line width=1pt] coordinates {
    (Junior,25000) (Mid,45000) (Senior,70000) (Lead,90000) (Architect,110000)
};

% Brasil
\addplot[fill=blue!70, draw=blue!90, line width=1pt] coordinates {
    (Junior,30000) (Mid,50000) (Senior,75000) (Lead,95000) (Architect,115000)
};

% Argentina
\addplot[fill=cyan!70, draw=cyan!90, line width=1pt] coordinates {
    (Junior,20000) (Mid,35000) (Senior,55000) (Lead,70000) (Architect,85000)
};

% Colombia
\addplot[fill=orange!70, draw=orange!90, line width=1pt] coordinates {
    (Junior,22000) (Mid,38000) (Senior,60000) (Lead,75000) (Architect,90000)
};

% Chile
\addplot[fill=purple!70, draw=purple!90, line width=1pt] coordinates {
    (Junior,28000) (Mid,48000) (Senior,72000) (Lead,88000) (Architect,105000)
};

\end{axis}
\end{tikzpicture}
\caption{Comparativa salarial DevOps en principales mercados LATAM}
\end{figure}

\vspace{1cm}

\subsection{Habilidades Más Demandadas 2024}
\begin{itemize}
    \item \textbf{Kubernetes:} 89\% de ofertas laborales
    \item \textbf{AWS:} 76\% de ofertas laborales
    \item \textbf{Terraform:} 68\% de ofertas laborales
    \item \textbf{Docker:} 82\% de ofertas laborales
    \item \textbf{Python:} 71\% de ofertas laborales
    \item \textbf{GitOps:} 45\% de ofertas laborales (tendencia creciente)
\end{itemize}

\section{Playbook de Resolución de Problemas DevOps}

\subsection{Incident Response Framework}
\begin{enumerate}
    \item \textbf{Detección:} Alertas automáticas o reporte manual
    \item \textbf{Triaje:} Evaluación de severidad e impacto
    \item \textbf{Escalación:} Notificación a equipos apropiados
    \item \textbf{Investigación:} Root cause analysis
    \item \textbf{Mitigación:} Acciones para restaurar servicio
    \item \textbf{Resolución:} Fix permanente del problema
    \item \textbf{Post-mortem:} Documentación y lecciones aprendidas
\end{enumerate}

\subsection{Post-Mortem Template}
\begin{lstlisting}[caption=Template de Post-Mortem DevOps]
# Post-Mortem: [Incident Title]

## Resumen Ejecutivo
- **Fecha:** [YYYY-MM-DD]
- **Duración:** [X horas Y minutos]
- **Impacto:** [Descripción del impacto en usuarios/negocio]
- **Root Cause:** [Causa raíz identificada]

## Timeline
- **HH:MM** - Inicio del incidente
- **HH:MM** - Primera detección
- **HH:MM** - Escalación a equipo DevOps
- **HH:MM** - Identificación de causa raíz
- **HH:MM** - Implementación de fix
- **HH:MM** - Servicio completamente restaurado

## Análisis de Causa Raíz
### ¿Qué salió mal?
[Descripción técnica detallada]

### ¿Por qué no se detectó antes?
[Análisis de gaps en monitoreo/alertas]

### ¿Cómo se puede prevenir?
[Medidas preventivas específicas]

## Action Items
- [ ] [Acción 1] - Responsable: [Nombre] - Fecha: [YYYY-MM-DD]
- [ ] [Acción 2] - Responsable: [Nombre] - Fecha: [YYYY-MM-DD]
- [ ] [Acción 3] - Responsable: [Nombre] - Fecha: [YYYY-MM-DD]

## Lecciones Aprendidas
1. [Lección 1]
2. [Lección 2]
3. [Lección 3]
\end{lstlisting}

\section{Troubleshooting y Debugging}

\subsection{Problemas Comunes en CI/CD}
\begin{itemize}
    \item \textbf{Build lento:} Paralelización, cache de dependencias, build incremental
    \item \textbf{Tests flaky:} Aislamiento, datos de prueba, timeouts apropiados
    \item \textbf{Despliegues fallidos:} Health checks, rollback automático, blue-green
    \item \textbf{Secrets management:} Vault, sealed secrets, rotación automática
\end{itemize}

\subsection{Debugging en Producción}
\begin{itemize}
    \item \textbf{Logs estructurados:} JSON, correlation IDs, niveles apropiados
    \item \textbf{Métricas en tiempo real:} RED method (Rate, Errors, Duration)
    \item \textbf{Distributed tracing:} Seguimiento de requests complejos
    \item \textbf{Profiling:} CPU, memoria, I/O bottlenecks
\end{itemize}

\subsection{Herramientas de Debugging}
\begin{itemize}
    \item \textbf{kubectl:} Debugging de Kubernetes pods y servicios
    \item \textbf{docker logs:} Logs de contenedores en tiempo real
    \item \textbf{curl/httpie:} Testing de APIs y endpoints
    \item \textbf{jq:} Parsing de JSON logs y responses
    \item \textbf{tcpdump/wireshark:} Análisis de tráfico de red
\end{itemize}

\section{Métricas y KPIs en DevOps}
Para medir el éxito de la implementación de DevOps, se utilizan métricas clave:

\begin{table}[h!]
\centering
\begin{tabular}{|l|l|p{6cm}|}
\hline
\textbf{Métrica} & \textbf{Objetivo} & \textbf{Descripción} \\
\hline
Lead Time & Reducir & Tiempo desde commit hasta producción \\
\hline
Deployment Frequency & Aumentar & Frecuencia de despliegues exitosos \\
\hline
MTTR & Reducir & Tiempo medio de recuperación ante fallos \\
\hline
Change Failure Rate & Reducir & Porcentaje de despliegues que causan problemas \\
\hline
\end{tabular}
\caption{Métricas principales de DevOps (DORA Metrics)}
\end{table}

\subsection{Visualización: Evolución de Métricas DORA}
\begin{figure}[h!]
\centering
\begin{tikzpicture}
\begin{axis}[
    title={Evolución de Lead Time por Nivel de Madurez DevOps},
    xlabel={Nivel de Madurez},
    ylabel={Lead Time (días)},
    xmin=0, xmax=4,
    ymin=0, ymax=200,
    xtick={1,2,3,4},
    xticklabels={Bajo,Medio,Alto,Elite},
    ytick={0,50,100,150,200},
    legend pos=north east,
    ymajorgrids=true,
    grid style=dashed,
    width=12cm,
    height=8cm
]

\addplot[
    color=red,
    mark=square,
    thick
    ]
    coordinates {
    (1,180)(2,30)(3,7)(4,1)
    };
\addlegendentry{Lead Time}

\addplot[
    color=blue,
    mark=triangle,
    thick
    ]
    coordinates {
    (1,45)(2,15)(3,5)(4,0.5)
    };
\addlegendentry{MTTR (horas)}

\end{axis}
\end{tikzpicture}
\caption{Reducción dramática de Lead Time y MTTR con madurez DevOps}
\end{figure}

\subsection{Gráfico: Frecuencia de Despliegue por Industria}
\begin{figure}[h!]
\centering
\begin{tikzpicture}
\begin{axis}[
    title={Frecuencia de Despliegue por Sector (despliegues/día)},
    xlabel={Sector},
    ylabel={Despliegues por día},
    ybar,
    bar width=20pt,
    width=12cm,
    height=8cm,
    ymin=0,
    symbolic x coords={Fintech,E-commerce,SaaS,Telecom,Gobierno},
    xtick=data,
    x tick label style={rotate=45,anchor=east},
    nodes near coords,
    nodes near coords align={vertical},
]

\addplot coordinates {
    (Fintech,25)
    (E-commerce,40) 
    (SaaS,60)
    (Telecom,15)
    (Gobierno,2)
};

\end{axis}
\end{tikzpicture}
\caption{Variación de frecuencia de despliegue según el sector industrial}
\end{figure}

\subsection{Comparativa: Antes vs Después de DevOps}
\begin{figure}[h!]
\centering
\begin{tikzpicture}
\begin{axis}[
    title={Impacto de DevOps: Métricas Antes vs Después},
    xlabel={Métricas},
    ylabel={Mejora (\%)},
    ybar,
    bar width=25pt,
    width=12cm,
    height=8cm,
    ymin=0,
    ymax=100,
    symbolic x coords={Lead Time,Deploy Freq,MTTR,Quality},
    xtick=data,
    x tick label style={rotate=45,anchor=east},
    nodes near coords,
    nodes near coords align={vertical},
    legend pos=north west,
]

\addplot[fill=green!50] coordinates {
    (Lead Time,85)
    (Deploy Freq,90) 
    (MTTR,75)
    (Quality,60)
};
\addlegendentry{Mejora promedio}

\end{axis}
\end{tikzpicture}
\caption{Mejoras típicas obtenidas con implementación DevOps exitosa}
\end{figure}

\section{Desafíos y Consideraciones}

\subsection{Desafíos Técnicos}
\begin{itemize}
    \item \textbf{Complejidad de herramientas:} Curva de aprendizaje empinada
    \item \textbf{Integración de sistemas:} Compatibilidad entre herramientas legacy y modernas
    \item \textbf{Gestión de configuración:} Mantener consistencia entre entornos
    \item \textbf{Escalabilidad:} Adaptar procesos a equipos y proyectos grandes
\end{itemize}

\subsection{Desafíos Organizacionales}
\begin{itemize}
    \item \textbf{Resistencia al cambio:} Transformación cultural organizacional
    \item \textbf{Silos departamentales:} Romper barreras entre equipos
    \item \textbf{Inversión inicial:} Costos de herramientas y capacitación
    \item \textbf{Métricas y KPIs:} Definir indicadores de éxito apropiados
\end{itemize}

\subsection{Mejores Prácticas para la Adopción}
\begin{itemize}
    \item \textbf{Comenzar pequeño:} Proyectos piloto antes de adopción masiva
    \item \textbf{Capacitación continua:} Inversión en formación del equipo
    \item \textbf{Automatización gradual:} Implementación incremental de procesos
    \item \textbf{Cultura de experimentación:} Fomentar el aprendizaje de fallos
\end{itemize}

\section{Tendencias Futuras en DevOps}

\subsection{Tecnologías Emergentes}
\begin{itemize}
    \item \textbf{GitOps:} Gestión declarativa de infraestructura mediante Git
    \item \textbf{Serverless/FaaS:} Functions as a Service para arquitecturas event-driven
    \item \textbf{Edge Computing:} Despliegues distribuidos cerca del usuario final
    \item \textbf{AI/ML Ops:} Integración de machine learning en pipelines DevOps
\end{itemize}

\subsection{Evolución de Prácticas}
\begin{itemize}
    \item \textbf{Platform Engineering:} Equipos dedicados a crear plataformas internas
    \item \textbf{Chaos Engineering:} Pruebas de resiliencia en producción
    \item \textbf{FinOps:} Optimización de costos en la nube
    \item \textbf{Green DevOps:} Prácticas sostenibles y eficiencia energética
\end{itemize}

\subsection{Impacto de la Inteligencia Artificial}
\begin{itemize}
    \item \textbf{Automated Testing:} IA para generación automática de pruebas
    \item \textbf{Predictive Analytics:} Predicción de fallos y optimización proactiva
    \item \textbf{Intelligent Monitoring:} Detección automática de anomalías
    \item \textbf{Code Generation:} Asistentes IA para desarrollo y configuración
\end{itemize}

\section{GitOps y Continuous Deployment Avanzado}

\subsection{GitOps: La Evolución del CD}
GitOps representa la evolución natural de Continuous Deployment:

\subsubsection{Principios GitOps}
\begin{itemize}
    \item \textbf{Git como fuente de verdad:} Toda la configuración versionada en Git
    \item \textbf{Declarativo:} Estado deseado definido en manifiestos
    \item \textbf{Automatización:} Reconciliación automática del estado
    \item \textbf{Observabilidad:} Visibilidad completa del estado del sistema
\end{itemize}

\subsection{ArgoCD: GitOps en Kubernetes}
ArgoCD es la herramienta líder para GitOps en Kubernetes:

\subsubsection{Arquitectura ArgoCD}
\begin{figure}[H]
\centering
\begin{tikzpicture}[
    node distance=1.8cm,
    repo/.style={rectangle, draw, fill=green!20, text width=2.2cm, text centered, minimum height=1.2cm, font=\footnotesize},
    argo/.style={rectangle, draw, fill=blue!20, text width=2.5cm, text centered, minimum height=1.2cm, font=\footnotesize},
    k8s/.style={rectangle, draw, fill=orange!20, text width=2.2cm, text centered, minimum height=1.2cm, font=\footnotesize},
    arrow/.style={->, thick, blue}
]

% Git Repository
\node[repo] (gitrepo) {Git\\Repository\\(Manifests)};

% ArgoCD Components
\node[argo, below=2cm of gitrepo] (argoserver) {ArgoCD\\Server};
\node[argo, left=of argoserver] (argorepo) {Repo\\Server};
\node[argo, right=of argoserver] (argocontroller) {Application\\Controller};

% Kubernetes Cluster
\node[k8s, below=2cm of argoserver] (k8sapi) {Kubernetes\\API Server};
\node[k8s, left=of k8sapi] (k8snode1) {Worker\\Node 1};
\node[k8s, right=of k8sapi] (k8snode2) {Worker\\Node 2};

% Connections
\draw[arrow] (gitrepo) -- (argorepo);
\draw[arrow] (argorepo) -- (argoserver);
\draw[arrow] (argoserver) -- (argocontroller);
\draw[arrow] (argocontroller) -- (k8sapi);
\draw[arrow] (k8sapi) -- (k8snode1);
\draw[arrow] (k8sapi) -- (k8snode2);

% Sync loop
\draw[arrow, dashed, red] (argocontroller) to[bend left=45] node[midway, right, font=\tiny] {Sync} (gitrepo);

% Labels
\node[above=0.5cm of gitrepo, font=\Large\bfseries] {ArgoCD GitOps Architecture};
\node[left=1cm of argorepo, font=\small, text=blue!70!black] {ArgoCD};
\node[left=1cm of k8snode1, font=\small, text=orange!70!black] {Kubernetes};

\end{tikzpicture}
\caption{Arquitectura ArgoCD para GitOps en Kubernetes}
\end{figure}

\newpage

\subsubsection{Configuración ArgoCD}
Ejemplo de Application manifest para ArgoCD:
\begin{lstlisting}[language=yaml, caption=ArgoCD Application Configuration]
apiVersion: argoproj.io/v1alpha1
kind: Application
metadata:
  name: devops-app
  namespace: argocd
  finalizers:
    - resources-finalizer.argocd.argoproj.io
spec:
  project: default
  source:
    repoURL: https://github.com/company/devops-app-config
    targetRevision: HEAD
    path: k8s/overlays/production
  destination:
    server: https://kubernetes.default.svc
    namespace: production
  syncPolicy:
    automated:
      prune: true
      selfHeal: true
      allowEmpty: false
    syncOptions:
    - CreateNamespace=true
    - PrunePropagationPolicy=foreground
    retry:
      limit: 5
      backoff:
        duration: 5s
        factor: 2
        maxDuration: 3m
  revisionHistoryLimit: 10
\end{lstlisting}

\section{DevOps en la Era de la Inteligencia Artificial}

\subsection{AI-Driven DevOps (AIOps)}
La integración de IA en DevOps está revolucionando las operaciones:

\subsubsection{Capacidades de AIOps}
\begin{itemize}
    \item \textbf{Detección predictiva:} Identificación de fallos antes de que ocurran
    \item \textbf{Auto-remediación:} Corrección automática de problemas conocidos
    \item \textbf{Optimización de recursos:} Ajuste dinámico basado en patrones de uso
    \item \textbf{Análisis de anomalías:} Detección de comportamientos inusuales en tiempo real
\end{itemize}

\subsection{Diagrama: AIOps en el Pipeline DevOps}
\begin{figure}[h!]
\centering
\begin{tikzpicture}[
    node distance=1.5cm,
    aibox/.style={rectangle, draw, fill=cyan!20, text width=2cm, text centered, minimum height=1cm, font=\footnotesize},
    pipeline/.style={rectangle, draw, fill=blue!15, text width=2cm, text centered, minimum height=1cm, font=\footnotesize},
    arrow/.style={->, thick, blue},
    aiflow/.style={->, thick, cyan, dashed}
]

% Traditional pipeline
\node[pipeline] (code) {Code\\Commit};
\node[pipeline, right=of code] (build) {Build};
\node[pipeline, right=of build] (test) {Test};
\node[pipeline, right=of test] (deploy) {Deploy};
\node[pipeline, right=of deploy] (monitor) {Monitor};

% AI layer
\node[aibox, above=2cm of build] (predict) {Predictive\\Analytics};
\node[aibox, above=2cm of test] (autotest) {AI Test\\Generation};
\node[aibox, above=2cm of deploy] (optimize) {Deployment\\Optimization};
\node[aibox, above=2cm of monitor] (anomaly) {Anomaly\\Detection};

% Pipeline flow
\draw[arrow] (code) -- (build);
\draw[arrow] (build) -- (test);
\draw[arrow] (test) -- (deploy);
\draw[arrow] (deploy) -- (monitor);

% AI interactions
\draw[aiflow] (predict) -- (build);
\draw[aiflow] (autotest) -- (test);
\draw[aiflow] (optimize) -- (deploy);
\draw[aiflow] (anomaly) -- (monitor);

% Feedback
\draw[aiflow] (anomaly) to[bend left=60] (predict);

% Labels
\node[above=1cm of autotest, font=\Large\bfseries] {AIOps Pipeline};
\node[below=0.5cm of test, font=\small] {Traditional DevOps};
\node[above=0.5cm of autotest, font=\small, text=cyan!70!black] {AI Enhancement Layer};

\end{tikzpicture}
\caption{Integración de AIOps en el pipeline DevOps tradicional}
\end{figure}

\section{Sostenibilidad y Green DevOps}

\subsection{Principios de Green DevOps}
\begin{itemize}
    \item \textbf{Eficiencia energética:} Optimización del consumo de recursos
    \item \textbf{Carbon footprint:} Medición y reducción de emisiones
    \item \textbf{Recursos compartidos:} Maximización del uso de infraestructura
    \item \textbf{Lifecycle management:} Gestión sostenible del ciclo de vida del software
\end{itemize}

\subsection{Métricas de Sostenibilidad}
\begin{table}[h!]
\centering
\begin{tabular}{|l|l|l|}
\hline
\textbf{Métrica} & \textbf{Unidad} & \textbf{Objetivo} \\
\hline
CPU Utilization & \% & > 70\% \\
\hline
Memory Efficiency & \% & > 80\% \\
\hline
Carbon Intensity & gCO2/kWh & < 100 \\
\hline
Resource Waste & \% & < 15\% \\
\hline
\end{tabular}
\caption{Métricas clave para Green DevOps}
\end{table}

\section{Quantum DevOps: Preparación para la Computación Cuántica}

\subsection{Introducción a Quantum DevOps}
La computación cuántica representa el próximo salto tecnológico que requerirá nuevos paradigmas DevOps:

\subsubsection{Desafíos Únicos del Quantum Computing}
\begin{itemize}
    \item \textbf{Coherencia cuántica:} Estados frágiles que requieren condiciones específicas
    \item \textbf{Error rates:} Tasas de error significativamente más altas que sistemas clásicos
    \item \textbf{Simulación limitada:} Imposibilidad de simular sistemas cuánticos grandes
    \item \textbf{Hardware especializado:} Dependencia de hardware cuántico específico
\end{itemize}

\subsection{Pipeline CI/CD para Algoritmos Cuánticos}
\begin{figure}[h!]
\centering
\begin{tikzpicture}[
    node distance=1.5cm,
    quantum/.style={rectangle, draw, fill=purple!20, text width=2.2cm, text centered, minimum height=1cm, font=\footnotesize},
    classical/.style={rectangle, draw, fill=blue!15, text width=2.2cm, text centered, minimum height=1cm, font=\footnotesize},
    arrow/.style={->, thick, purple}
]

% Quantum Pipeline
\node[quantum] (qcode) {Quantum\\Code\\(Qiskit/Cirq)};
\node[quantum, right=of qcode] (qsim) {Quantum\\Simulator\\Testing};
\node[quantum, right=of qsim] (qopt) {Circuit\\Optimization};
\node[quantum, right=of qopt] (qhw) {Quantum\\Hardware\\Deploy};

% Classical Support
\node[classical, below=2cm of qcode] (ccode) {Classical\\Control\\Code};
\node[classical, below=2cm of qsim] (ctest) {Integration\\Testing};
\node[classical, below=2cm of qopt] (cmon) {Quantum\\Monitoring};
\node[classical, below=2cm of qhw] (cpost) {Result\\Processing};

% Quantum flow
\draw[arrow] (qcode) -- (qsim);
\draw[arrow] (qsim) -- (qopt);
\draw[arrow] (qopt) -- (qhw);

% Classical flow
\draw[arrow] (ccode) -- (ctest);
\draw[arrow] (ctest) -- (cmon);
\draw[arrow] (cmon) -- (cpost);

% Integration
\draw[arrow, dashed] (qcode) -- (ccode);
\draw[arrow, dashed] (qsim) -- (ctest);
\draw[arrow, dashed] (qopt) -- (cmon);
\draw[arrow, dashed] (qhw) -- (cpost);

% Labels
\node[above=1cm of qsim, font=\Large\bfseries] {Quantum DevOps Pipeline};
\node[left=0.5cm of qcode, font=\small, text=purple!70!black] {Quantum};
\node[left=0.5cm of ccode, font=\small, text=blue!70!black] {Classical};

\end{tikzpicture}
\caption{Pipeline DevOps híbrido para computación cuántica}
\end{figure}

\subsection{Herramientas Quantum DevOps}
\begin{itemize}
    \item \textbf{Qiskit:} Framework de IBM para desarrollo cuántico
    \item \textbf{Cirq:} Biblioteca de Google para circuitos cuánticos
    \item \textbf{PennyLane:} Machine learning cuántico diferenciable
    \item \textbf{Forest:} Plataforma de Rigetti para computación cuántica
    \item \textbf{Q\#:} Lenguaje de Microsoft para desarrollo cuántico
\end{itemize}

\subsection{Testing en Entornos Cuánticos}
\begin{lstlisting}[language=python, caption=Ejemplo de testing cuántico con Qiskit]
import unittest
from qiskit import QuantumCircuit, Aer, execute
from qiskit.quantum_info import Statevector

class QuantumAlgorithmTest(unittest.TestCase):
    
    def setUp(self):
        self.backend = Aer.get_backend('statevector_simulator')
        
    def test_bell_state_preparation(self):
        """Test Bell state preparation circuit"""
        qc = QuantumCircuit(2)
        qc.h(0)  # Hadamard gate on qubit 0
        qc.cx(0, 1)  # CNOT gate
        
        # Execute circuit
        job = execute(qc, self.backend)
        result = job.result()
        statevector = result.get_statevector()
        
        # Expected Bell state: |00⟩ + |11⟩ (normalized)
        expected = Statevector.from_label('00') + Statevector.from_label('11')
        expected = expected / expected.norm()
        
        # Assert states are equivalent (within tolerance)
        self.assertTrue(statevector.equiv(expected))
        
    def test_quantum_error_correction(self):
        """Test quantum error correction code"""
        # Implementation of error correction testing
        pass
        
    def test_quantum_algorithm_performance(self):
        """Test algorithm performance on different backends"""
        backends = ['qasm_simulator', 'statevector_simulator']
        
        for backend_name in backends:
            with self.subTest(backend=backend_name):
                backend = Aer.get_backend(backend_name)
                # Performance testing logic
                pass

if __name__ == '__main__':
    unittest.main()
\end{lstlisting}

\section{DevOps en el Ecosistema Web3 y Blockchain}

\subsection{Desafíos DevOps en Blockchain}
\begin{itemize}
    \item \textbf{Inmutabilidad:} Los smart contracts no se pueden modificar una vez desplegados
    \item \textbf{Gas costs:} Cada operación tiene un costo en la blockchain
    \item \textbf{Múltiples redes:} Despliegue en mainnet, testnets, y sidechains
    \item \textbf{Seguridad crítica:} Vulnerabilidades pueden resultar en pérdidas financieras
\end{itemize}

\subsection{Pipeline CI/CD para Smart Contracts}
\begin{lstlisting}[language=javascript, caption=Hardhat deployment script para smart contracts]
// deploy/01-deploy-contract.js
const { network } = require("hardhat")
const { developmentChains, networkConfig } = require("../helper-hardhat-config")
const { verify } = require("../utils/verify")

module.exports = async ({ getNamedAccounts, deployments }) => {
    const { deploy, log } = deployments
    const { deployer } = await getNamedAccounts()
    
    const args = [
        networkConfig[network.config.chainId]["initialSupply"],
        networkConfig[network.config.chainId]["tokenName"]
    ]
    
    log("----------------------------------------------------")
    log("Deploying DevOpsToken and waiting for confirmations...")
    
    const devOpsToken = await deploy("DevOpsToken", {
        from: deployer,
        args: args,
        log: true,
        waitConfirmations: network.config.blockConfirmations || 1,
    })
    
    // Verify contract on Etherscan if not on development chain
    if (!developmentChains.includes(network.name) && process.env.ETHERSCAN_API_KEY) {
        log("Verifying contract on Etherscan...")
        await verify(devOpsToken.address, args)
    }
    
    log("----------------------------------------------------")
}

module.exports.tags = ["all", "devopstoken"]
\end{lstlisting}

\subsection{Testing de Smart Contracts}
\begin{lstlisting}[language=javascript, caption=Tests automatizados para smart contracts]
const { expect } = require("chai")
const { ethers, deployments } = require("hardhat")

describe("DevOpsToken", function () {
    let devOpsToken, deployer, user1
    
    beforeEach(async function () {
        const accounts = await ethers.getSigners()
        deployer = accounts[0]
        user1 = accounts[1]
        
        await deployments.fixture(["all"])
        devOpsToken = await ethers.getContract("DevOpsToken", deployer)
    })
    
    describe("Deployment", function () {
        it("Should set the right owner", async function () {
            expect(await devOpsToken.owner()).to.equal(deployer.address)
        })
        
        it("Should assign total supply to owner", async function () {
            const ownerBalance = await devOpsToken.balanceOf(deployer.address)
            expect(await devOpsToken.totalSupply()).to.equal(ownerBalance)
        })
    })
    
    describe("Transactions", function () {
        it("Should transfer tokens between accounts", async function () {
            await devOpsToken.transfer(user1.address, 50)
            expect(await devOpsToken.balanceOf(user1.address)).to.equal(50)
        })
        
        it("Should fail if sender doesn't have enough tokens", async function () {
            const initialOwnerBalance = await devOpsToken.balanceOf(deployer.address)
            
            await expect(
                devOpsToken.connect(user1).transfer(deployer.address, 1)
            ).to.be.revertedWith("ERC20: transfer amount exceeds balance")
            
            expect(await devOpsToken.balanceOf(deployer.address)).to.equal(
                initialOwnerBalance
            )
        })
    })
    
    describe("Security", function () {
        it("Should prevent reentrancy attacks", async function () {
            // Reentrancy attack testing logic
        })
        
        it("Should handle integer overflow/underflow", async function () {
            // Overflow/underflow testing logic
        })
    })
})
\end{lstlisting}

\section{DevOps para Edge Computing}

\subsection{Desafíos del Edge}
\begin{itemize}
    \item \textbf{Latencia ultra-baja:} < 10ms para aplicaciones críticas
    \item \textbf{Recursos limitados:} Optimización para hardware restringido
    \item \textbf{Conectividad intermitente:} Operación offline y sincronización
    \item \textbf{Gestión distribuida:} Miles de nodos edge simultáneos
\end{itemize}

\subsection{Arquitectura Edge DevOps}
\begin{figure}[h!]
\centering
\begin{tikzpicture}[
    node distance=1.5cm,
    cloud/.style={ellipse, draw, fill=blue!20, text width=2.5cm, text centered, minimum height=1cm, font=\footnotesize},
    edge/.style={rectangle, draw, fill=green!20, text width=1.8cm, text centered, minimum height=0.8cm, font=\tiny},
    device/.style={circle, draw, fill=orange!20, text width=1cm, text centered, font=\tiny}
]

% Cloud layer
\node[cloud] (cloudci) {Cloud\\CI/CD\\Pipeline};

% Edge layer
\node[edge, below left=2cm and 1cm of cloudci] (edge1) {Edge Node\\Americas};
\node[edge, below=2cm of cloudci] (edge2) {Edge Node\\Europe};
\node[edge, below right=2cm and 1cm of cloudci] (edge3) {Edge Node\\Asia};

% Device layer
\node[device, below=1.5cm of edge1] (dev1) {IoT\\Device};
\node[device, below=1.5cm of edge2] (dev2) {Mobile\\App};
\node[device, below=1.5cm of edge3] (dev3) {Sensor\\Network};

% Connections
\draw[->] (cloudci) -- (edge1);
\draw[->] (cloudci) -- (edge2);
\draw[->] (cloudci) -- (edge3);

\draw[->] (edge1) -- (dev1);
\draw[->] (edge2) -- (dev2);
\draw[->] (edge3) -- (dev3);

% Labels
\node[above=0.5cm of cloudci, font=\Large\bfseries] {Edge DevOps Architecture};
\node[left=0.5cm of edge1, font=\small] {Edge Layer};
\node[left=0.5cm of dev1, font=\small] {Device Layer};

\end{tikzpicture}
\caption{Arquitectura DevOps distribuida para Edge Computing}
\end{figure}

\section{Conclusión}
DevOps representa un cambio cultural y técnico fundamental en el desarrollo de software moderno. Su implementación exitosa requiere compromiso organizacional, automatización de procesos y una mentalidad de mejora continua. Las organizaciones que adoptan DevOps logran entregar software de alta calidad de manera más rápida y confiable, mejorando significativamente la colaboración entre equipos de desarrollo y operaciones.

La medición constante a través de métricas DORA y la adaptación continua son elementos clave para el éxito a largo plazo de cualquier iniciativa DevOps.

\subsection{El Futuro de DevOps}
Mirando hacia adelante, DevOps continuará evolucionando con:
\begin{itemize}
    \item \textbf{Integración de IA:} AIOps transformará la operación y mantenimiento
    \item \textbf{Sostenibilidad:} Green DevOps será un requisito, no una opción
    \item \textbf{Edge Computing:} Nuevos paradigmas de despliegue distribuido
    \item \textbf{Quantum Computing:} Preparación para la próxima revolución tecnológica
\end{itemize}

\newpage
% --- Bibliografía ---
\begin{thebibliography}{15}
\bibitem{kim} 
Gene Kim, Jez Humble, Patrick Debois, John Willis. 
\textit{The DevOps Handbook: How to Create World-Class Agility, Reliability, and Security in Technology Organizations}. IT Revolution Press, 2016.

\bibitem{humble} 
Jez Humble, David Farley. 
\textit{Continuous Delivery: Reliable Software Releases through Build, Test, and Deployment Automation}. Addison-Wesley, 2010.

\bibitem{bass} 
Len Bass, Ingo Weber, Liming Zhu. 
\textit{DevOps: A Software Architect's Perspective}. Addison-Wesley, 2015.

\bibitem{forsgren} 
Nicole Forsgren, Jez Humble, Gene Kim. 
\textit{Accelerate: The Science of Lean Software and DevOps}. IT Revolution Press, 2018.

\bibitem{dora2023} 
DORA Team. 
\textit{2023 State of DevOps Report}. Google Cloud, 2023.

\bibitem{kubernetes} 
Brendan Burns, Joe Beda, Kelsey Hightower. 
\textit{Kubernetes: Up and Running}. O'Reilly Media, 2019.

\bibitem{terraform} 
Yevgeniy Brikman. 
\textit{Terraform: Up \& Running}. O'Reilly Media, 2022.

\bibitem{sre} 
Betsy Beyer, Chris Jones, Jennifer Petoff, Niall Richard Murphy. 
\textit{Site Reliability Engineering: How Google Runs Production Systems}. O'Reilly Media, 2016.

\bibitem{phoenix} 
Gene Kim, Kevin Behr, George Spafford. 
\textit{The Phoenix Project: A Novel about IT, DevOps, and Helping Your Business Win}. IT Revolution Press, 2018.

\bibitem{aws} 
Amazon Web Services. 
\textit{What is DevOps?} \url{https://aws.amazon.com/devops/what-is-devops/}

\bibitem{cncf} 
Cloud Native Computing Foundation. 
\textit{CNCF Cloud Native Interactive Landscape}. \url{https://landscape.cncf.io/}

\bibitem{puppet} 
Puppet Labs. 
\textit{2023 State of DevOps Report}. \url{https://puppet.com/resources/state-of-devops-report}

\bibitem{atlassian} 
Atlassian. 
\textit{DevOps Best Practices and Tools}. \url{https://www.atlassian.com/devops}

\bibitem{redhat} 
Red Hat. 
\textit{What is DevOps?} \url{https://www.redhat.com/en/topics/devops}

\bibitem{microsoft} 
Microsoft Azure. 
\textit{What is DevOps? DevOps Explained}. \url{https://azure.microsoft.com/en-us/resources/cloud-computing-dictionary/what-is-devops/}

\bibitem{mercadolibre} 
MercadoLibre Engineering Team. 
\textit{Scaling DevOps in Latin America: Lessons from MercadoLibre}. Engineering Blog, 2023.

\bibitem{rappi} 
Rappi Engineering. 
\textit{Building a DevOps Culture in a Fast-Growing Startup}. Medium Engineering, 2023.

\bibitem{bancoazteca} 
Banco Azteca IT. 
\textit{Digital Transformation in Mexican Banking: A DevOps Journey}. Financial Technology Conference, 2023.

\bibitem{devopsmx} 
DevOps México Community. 
\textit{Estado del DevOps en México 2023}. \url{https://devops.mx/estado-devops-2023}

\bibitem{latamdevops} 
LATAM DevOps Survey. 
\textit{DevOps Adoption in Latin America: Trends and Challenges}. 2023.

\bibitem{argocd} 
Argo Project. 
\textit{ArgoCD: Declarative GitOps CD for Kubernetes}. \url{https://argo-cd.readthedocs.io/}

\bibitem{qiskit} 
IBM Quantum Team. 
\textit{Qiskit: An Open-source Framework for Quantum Computing}. Nature, 2021.

\bibitem{nubank} 
Nubank Engineering. 
\textit{Building a Digital Bank: Lessons from Nubank's DevOps Journey}. InfoQ, 2023.

\bibitem{kavak} 
Kavak Engineering Team. 
\textit{Scaling Across LATAM: DevOps at Kavak}. Medium Engineering, 2023.

\bibitem{govmx} 
Gobierno de México. 
\textit{Estrategia Digital Nacional 2021-2024}. Coordinación de Estrategia Digital Nacional, 2021.

\bibitem{serpro} 
SERPRO Brasil. 
\textit{Modernização da Infraestrutura Federal: Caso SERPRO}. Congresso Brasileiro de Software Livre, 2023.

\bibitem{backstage} 
Spotify Engineering. 
\textit{Backstage: An Open Platform for Building Developer Portals}. Spotify Labs, 2020.

\bibitem{crossplane} 
Crossplane Community. 
\textit{Crossplane: The Cloud Native Control Plane}. CNCF Landscape, 2023.

\bibitem{quantumdevops} 
Chen, L., Wang, M., Zhang, Y. 
\textit{Quantum DevOps: Bridging Classical and Quantum Computing Paradigms}. IEEE Quantum Engineering, 2024.

\bibitem{web3devops} 
Ethereum Foundation. 
\textit{Smart Contract Development Best Practices}. Ethereum Developer Resources, 2023.

\bibitem{greendevops} 
Green Software Foundation. 
\textit{Sustainable Software Engineering: Principles and Practices}. GSF Guidelines, 2023.

\bibitem{aiops} 
Gartner Research. 
\textit{Market Guide for AIOps Platforms}. Gartner IT Operations Research, 2024.

\bibitem{devopssalary} 
Stack Overflow. 
\textit{Developer Survey 2024: DevOps and Platform Engineering}. Stack Overflow Insights, 2024.

\bibitem{dora2024} 
DORA Team. 
\textit{2024 State of DevOps Report: The Evolution Continues}. Google Cloud, 2024.
\end{thebibliography}

\newpage
\section*{Índice de Términos Técnicos}
\addcontentsline{toc}{section}{Índice de Términos Técnicos}
\begin{multicols}{2}
\printindex
\end{multicols}

\end{document}
